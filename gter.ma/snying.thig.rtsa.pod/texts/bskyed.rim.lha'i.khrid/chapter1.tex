
\Section*{}
\\
Кланяюсь славному победителю Ваджрасаттве!

\begin{verse}
Чистый потенциал — совершенно чистый по природе\\
И свобод\-ный от завес,\\
Удинство проявлений и пустоты,\\
Чудесные мани\-фестации Мудрости.\\
Дабы это состояние трех Тел было легко достигнуто,\\
Я объясню то, чем руководствуются\\
В медитации йоги божества.\\
\end{verse}
\\
Живые существа блуждают в обманывающей их Сансаре ('khrul ba'i 'khorlo)
вследствие того, что их просветленный Потенциал временно нечист.
Стадия Зарождения Кьерим практики йидама устанавливает это
состояние как просветлённые Тело, Речь и Ум татхагаты, тайное
и непостижимое измерение абсолютного Пространства. Здесь я
обещаю полностью раскрыть этутему, которая называется
"Лестница в Акаништху".\\
\\
Текст состоит из трех основных разделов:\\
\\
\begin{tabular}{ll}
1 & Безошибочная причина — основа вступления в Кьерим.\\
2 & Безошибочное условие — путь медитации.\\
3 & Чистый результат — способ достижения \\
  & состояния единства.\\
\end{tabular}
\newpage
\section{Безошибочная причина.\\Основа вступления в Стадию Зарождения}

\vspace{1cm}
В "Этапах Пути" (lam rim) говорится:

\begin{verse}
Обладающий сокровищем,\\
Совершенным потоком и интересом,\\
Знанием тантры и активности,\\
Сущностными наставлениями и теплом,\\
— Таков учитель, наделённый восемью качествами.
\end{verse}

Здесь указывается, что необходимо приступать к изучению Стадии Зарождения под
руководством Ваджрного Учителя, обладающего восемью характеристиками, и служить ему
(или ей) тремя способами. Следующий этап — созревание, посред\-ством получения всех
необходимых посвящений:\\
\\
\begin{tabular}{ll}
1 & Благоприятствующее внешнее посвящение (phyi phan pa);\\
2 & Внутреннее посвящение силы (nang nus pa);\\
3 & Глубокое тайное посвящение (gsan ba zab mo).\\
\end{tabular}
\\
\\
\\
С этого момента необходимо поддерживать различные обеты самаи и обязательства
(dam tshig dang sdom pa). А также следует заниматься практикой Стадии Зарождения йоги
божества.\\

\begin{siderules}
Эта часть содержит два раздела: 1) учения об очищении привычных тенденций, связанных с четырьмя видами рождения и
2) трех Самадхи — основе для процесса зарождения.\tablefootnote{Здесь и далее, выделенные абзацы
текста соответствуют комментарию Патрула Чокьи Вангпо,
который называется "Разъяснение трудных мест в Стадии зарождения и Йоге божества"
(bskyed rim lha'i khrid kyi dka' gnad cung zad bshad pa bzhugs).}
\end{siderules}
\\
\\
\newpage

\subsection{Четыре способа рождения и зарождения}
\vspace{1cm}
\begin{siderules}
Вцелом, ключевым моментом (смыслом) всех путей Великой колесницы является очищение
природы Сансары — Истины страдания и ее источника, а также привнесение
результи\-рующего Состояния Будды на путь (lam du byed ра). И хотя Истина страдания
проявляется в различных формах, все они укоренены в рождении (skye ba).
Далее, все формы страдания могут быть разделены на рождение и смерть.
Поэтому есть две стадии, которые очищают этот двойной процесс рождения и смерти:
Стадия Зарождения и Стадия Завершения. Вся текстовая традиция Тайной мантры
связана с этими двумя практиками.
\end{siderules}

В "Славной магической сети" (dpal sgyu 'phrul drwa bа)\\ сказано:\\

\begin{verse}
Чтобы очистить четыре способа рождения,\\
Имеются четыре способа зарождения:\\
Очень сложный и сложный,\\
Простой и очень простой.
\end{verse}

\begin{siderules}
Как здесь показано, есть четыре типа рождения. Эти четыре, в свою очередь, связаны с
четырьмя типами практики Стадии Зарождения: краткой, средней, простран\-ной и очень
пространной (bsdus 'brin rgyas pa dang rgyas pa).
\end{siderules}

\newpage
\subsubsection{Краткое описание предельной простоты}
\vspace{1cm}
Те, кто обладают наивысшей проницательностью (высшими способностями) (dbang ро
yang rab), практикуют ритуал Стадии Зарождения предельной простоты (shin tu spros pa med
pa) посредством развития внутренней энергии ума (rtsal sems), который связан с общим
воззрением (lta ba spyi) наивысшей царской естественной Колесницы (rang bzhin theg mchog
rgyal ро). И мощный импульс (btsan thabs), который возникает из этого процесса, позволяет
упражняться в нераздельности Стадий Зарождения и Завершения. Не опираясь на слова,
природа ума предстает в своём изначаль\-ном состоянии, как совершенная форма божества.
Это происходит подобно мгнове\-нному появлению отражения в зеркале. Следу\-ющая цитата
описывает этот подход.

\begin{verse}
Божество — ты, и ты — божество.\\
Ты и божество возникаете вместе.\\
Поскольку Самайя и Мудрость недвойственны,\\
Нет необходимости ни приглашать божество,\\
Ни просить его воссесть.\\
Эманирующее из себя, самопосвящающее,\\
И самоосознающее —это Три Корня.\\
\end{verse}

В этой форме Стадии Зарождения природа божества неотъем\-лемо и совершенным
образом присутствует в иллюзорном проявлении Мудрости (yeshes sgyu mа).
Это очищает "чудесное рождение (rdzus skyes)". Объясняя далее, всеведущий Лонгчен\-па пишет:

\begin{verse}
Также как чудесное рождение возникает мгновенно,\\
Нет необходимости начинать с самого начала,\\
А затем медитировать\\
На Стадиях Зарождения и Завершения.\\
\end{verse}

\newpage
\subsubsection{Промежуточный средний простой подход}
\vspace{1cm}
Истина недвойственности Мудрости и Пространства позволяет тем, ктоо бладает
наивысшей проницательностью (способно\-стью) (dbang ро rab tu), практиковать простую
Стадию Зарож\-дения, используя мгновенный подход (cig char du 'jug pa). В такой практике
божества становятся совершенными в резуль\-тате припоминанияих сущности (snying ро dran
ра), и они незарождаются посредством слов. Можно также практиковать постепенное
вхождение (rim gyis 'jug), при котором божества естественным образом проявляются в
открытой сфере Саманта\-бхадри, в великой пустоте Пространства мудрости (shes rab kyi
dbyings). Таким образом, принимают единую и недвойственную иллюзорную форму
божества. В "Тантре естественного возник\-новения ригпа" (rang shar) об этом говорится так:

\begin{verse}
Что есть мгновенная практика?\\
Божество незарождают,но оно становится совершенным\\
В момент воспоминания сущности.\\
Как практикуют постепенное вхождение?\\
Посредством постепенного вхождения\\
В Пространство и Мудрость.\\
\end{verse}

Как здесь отмечено, опора — чудесный дворец (gzhal yas khang) и «опирающееся» —
божественная природа (lha'i rang bzhin), проявляющиеся совершенным
образом, либо посред\-ством произнесенияих (божеств) имён, либо посредством
при\-поминания их сущности. Такой вид Стадии Зарождения очища\-ет привычные тенденции
(bag chags), связанные с рождением из тепла и влаги (drodg sher skyes), а также
предполагает узнавание нераздельности Блаженства и Пустоты (bdes tong).

\newpage
\subsubsection{Детальный и очень детальный подходы}
\vspace{1cm}
Очень детальная Стадия Зарождения очищает рождение из яйца (sgong skye) и
предназначена для тех, чей ум склонен к концептуализации (rtog bcas). В этой связи
причинный херука возникает как результирующий Ваджрадхара (rdo rje 'dzin ра), когда
медитируют на различных аспектах ребёнка-себя и ребёнка-других. В учениях раздела
"Садхан Великих Восьми [Кагье]" (sgrub sde chen ро bka' brgyad) говорится, что существует
пять шагов, когда "других делают своим сыном" (bdag gi sras su gzhan bya ba):\\
\\
\begin{tabular}{ll}
1 & Из причинного семенного слога свет распространяется \\
  & во вне и втягивается обратно (spro bsdus), \\
  & порождая первичных Супругов (yab yum);\\
2 & Будды десяти направлений притягиваются (bkug) \\
  & и растворяются (bstim ра) в пространстве;\\
3 & Собираются живые существа и их завесы (sgrib) очищаются;\\
4 & Провозглашается (brjod ра) величие недвойственности;\\
5 & Божества вытягиваются (появляются?) (bton) \\
  & из пространства и помещаются (dgod ра) в мандалу.\\
\end{tabular}

\begin{siderules}
Если краткую и среднюю стадии легко понять, то в очень пространной стадии практики,
описанной выше, мы встречаемся с такими темами как "наш собственный сын" и "сын
других". Следующий отрывок рассматривает эту тему с точки зрения раздела тантр
(rgyud sde). В отношении "своего собственного сына" в "Магической сети" сказано:
\begin{verse}
Зная,что сам Плод является путем,\\
Естественным и без противоречий,\\
Медитируй на всех мандалах и тигле\\
Без исключения, как на своем сыне.\\
\end{verse}
\end{siderules}

\newpage 
"Делать себя сыном других" (gzhan gyi sras su bdag bya ba) состоит из восьми частей:\\
\\
\begin{tabular}{ll}
1 & Первичные супруги растворяются (bzhu) \\
  & и превращаются в причинный семенной слог;\\
2 & Супруги зарождаются из этого семенного слога;\\
3 & Из концепций (rtog tshogs) супруга зарождаются слоги;\\
4 & Из супруги излучается свет и \\
  & призывает божество (gsolbagdabра);\\
5 & Все мандалы растворяются в себе как мужские \\
  & и женские супруги, порождая гордость божества \\
  & Мудрости (ye shes par nga rgyal);\\
6 & Супруги соединяются (sbyor ba), \\
  & и в пространстве порождается мандала;\\
7 & 42 вида собственных концепций (rtogра) превращаются\\
  &  в божеств и расходятся вовне (phyir gdon ра);\\
8 & Божества мудрости приглашаются (spyan drang), \\
  & запечатываются (rgyas gdab ра) и т.д.\\
\end{tabular}

\begin{siderules}
И в связи с "сыном других" сказано:
\begin{verse}
Если умы нераздельно вовлечены и слиты\\
Одним, обращенным с мольбой к другому,\\
Тогда какой смысл становиться сыном?\\
\end{verse}

Тогда как в "Тантре килы" (yang phurpa'i rgyud) относительно "сына других" сказано:
\begin{verse}
Подобно возникающему силой Будд,\\
В ритуале спонтанности\\
И из завершения ветвей ритуала,\\
Появляется мудрый — сын Будд.\\
\end{verse}
А также относительно "собственного сына" сказано:
\begin{verse}
Сын и царица возникают\\
Из явлений Сансары.\\
\end{verse}
Тогда как в разделе "Восьми великих садхан" сказано:
«Есть пять способов сделать других своим ребенком».\\

И хотя эта тема довольно ясная в каждом отдельном тексте, похоже, что нет ясного
объяснения, в котором было бы точно указано, что означает "собственный сын" и "сын
других". [Именно] Поэтому данная тема может казаться трудной для объяснения.
Однако, когда эта тема исследуется, первый и последний отрывок могут быть поняты как учащие
тому, что основа зарождения (bskyed gzhi) является Дхарматой самости — Сугатагарбхой.
Таким образом, визуализируя форму этого причинного Держате\-ля ваджры (Ваджрадхары),
"другие" — все цепляния и прояв\-ления (snang zheri), которые мы воспринимаем в Сансаре и
Нирване, вместе с промежуточным состоянием (bar srid) — собираются, а затем
"делают сына", извлекая его и злона. Этот процесс называется "делать других
своим собственным сыном".\\
\\
Когда "делают себя сыном других", то все проявления и восприятия Сансары и Нирваны
("другие") визуализируют\-ся как главное божество мандалы (Ваджрадхара). Это превра\-щает
все накопленные цепляния по отношению к скандхам, дхату и аятанам в семенной слог,
который затем становится "сделанным сыном" в результате извлечения из лона. Таким
образом становятся "сыном других".\\
\\
В среднем отрывке сказано, что сансарические существа являются полностью и совершенно
просветленными, как Будды трех времен. И здесь живые существа рассматрива\-ются как
"сами", а Будды как "другие". Делая себя "ребенком Будд", появляются в результате
зарождения себя как глав\-ного божества мандалы в ходе практик совершенствования,
таких как ритуал спонтанности (grub pa'i choga) и пять ветвей ритуала (cho ga'i yan lag lnga).\\
\newpage
"Делать других своим собственным ребенком", с другой стороны, относится к процессу,
в котором все проявляюще\-еся как собрание божеств в круге мандалы является "другим",
а все сансарические существа — "собой". После\-довательность, в которой каждый из них возникает,
гармонизируется (sgo bstun) в этом подходе, что ведет к союзу мужского и женского
супругов, а также к последую\-щему развитию через извлечение из лона. Первые два из этих
объяснений представляют эту концепцию с точки зрения основы зарож\-дения, тогда как
последующее — с точки зрения процесса зарождения (bskyed tshul).\\
\\
Здесь может возникнуть вопрос: являются ли эти два подхода представления себя сыновьями
хоть сколько-нибудь различными. Однако, в "Колеснице всеведущих" (rnam mkhyen shing
rta) есть объяснение, которое связывает все три. Поэтому я призываю разумных существ
тщательно исследовать это объяснение и открыть врата к этому ясному объяснению.
\end{siderules}
\vspace{0.2cm}
\\
Согласно тексту "Великая волшебная сеть" (rgyu 'phrul drwaba chenро) начинать
нужно с принятия Прибежища и зарождения Бодхичитты. Происходящее далее и связанное
с рождением из яйца, осуществляется в два этапа. Сначала, мгновенно представьте себя
изначальными супругами (yab yum), а затем пригласите в пространство перед собой мандалу,
на которую вы медитируете. Далее совершайте подношения, восхваления, молитвы,
постоянные раскаяния (mchod bstod gsol gdab rgyun bshags) и т.д. После этого используйте
ВАДЖРА МУ, чтобы снова пребывать в медитативном равновесии (mnyam par gzhag) в
сфере Пустоты. Этот последний пункт характерен исключительно для данного подхода.
Зная это, следующее представление может быть использовано как в предельно-развернутом
подходе, так и в средне детализированной Стадии Зарождения, которая очищает рождение
из утробы.\\
\\
В этих случаях основной упор делается на искусных методах Стадии Зарождения.
Также как буйствующий слон может быть управляем крюком или толпой, есть два подхода к этой
активности. Форма божественной обители и самих божеств очищают два элемента: опору и
то, что опирается. Первое — [опора] относится к внешней вселенной, а второе 
[то, что опирается] — к физическому телу. Существованиеих обоих 
[вселенная-«опора» и мы — существа-«опирающееся»] опирается на различные
привычные тенденции (bag chags). Следующий метод успокаивает склонность ума
испускать внешние объекты (yul la 'phro ba) и заключается в установлении
столба (rtod ра) глубокого медитативного погружения (ting ge 'dzin).\\
\\
Этот подход позволяет соединиться с сущностью очищения, совершенствования и
созревания (dag rdzogs smin). А точнее, поскольку это соответствует характеристикам
Сансары, существование (srid ра) очищается и улучшается. И поскольку это сходно с путём
Нирваны, результат совершенен в основе (gzhi la rdzogs). И, наконец, оба они приводят к
созреванию Стадии Завершения. Это общее понимание имеет чрезвычайно важное значение.\\

\begin{siderules}
Пространный способ зарождения предполагает зарожде\-ние пяти манифестаций просветления
(mngon byang). В данном контексте делается тройное разделение: 1) мани\-фестация
просветления основы; 2) манифестация просветления пути и 3) манифестация просветления
Плода. Первое связано с основой очищения, сущностной приро\-дой (rang bzhin ngo bo)
Сансары. Второе — с процессом очищения, где прилагают усилия на пути и зарождают это в
своем потоке [ума]. Третье  это результат очищения, состояние актуализации результата,
которое становится совершенным (mthar thug). И хотя есть три разделения, на самом деле это
ни что иное как Ясный свет ('od gsal) четырех пустотностей и сопровождающее их единство
(zung 'jug). Как таковые, они охватывают главные моменты всех путей Тайной Мантры.\\

Обычные тело, речь и ум, вместе с собранием (tshogs), или же белый и красный элементы, а
также праны — ум (rlung sems) вместе с собранием, являются изнанально просветленными
и имеющими сущность пяти мудростей (ye nas ye shes lnga'i ngo bor' sangs rgyas pa). Такова
манифестация просветления на этапе основы.\\

Символическая Мудрость (dpe'i yeshes) связана с путе мнакопления и путем соединения,
тогда как актуализация истинного Ясного света (don gyi 'od gsal mngon du byas pa)
предполагает четыре пустотности, сопровождающиеся единством (союзом), которые
возникают на пути видения и путимедитации. Эти факторы зарождаются в нашем потоке
(rgyud) через Стадии Зарождения и Завершения. В начале медитируют на пятивещах — луна,
солнце и т.д. Затем подключаются пять мудростей, связанных с проникновением ключевых
моментов (gnad du bsnan) белой и красной сущностей, а также ветра и ума. Этот процесс
предполагает манифестацию просветления на пути.\\
\\
Когда же достигают окончательного результата этого процесса, то белый элемент
естественным образом присут\-ствует как ваджрное тело, красный элемент — как ваджрная
речь, а осознающее сознание (shes rig) — как ваджрный ум. Таким образом, проявляются как
воплощение просветлен\-ного ума — Мудрость Дхармакаи, Ясный свет совершенного
результата. И Дхармакая, в свою очередь, не отдельна от Самбхогакаи, которая воплощает
океан знаков, черт и совершенных качеств (mtshan dpe yon tan rgya mtsho). Таково единство
за пределами обучения (mi slob ра 'zung 'jug), манифестация просветления совершенного
Плода.\\
\\
Основа очищения в этом процессе — это очень тонкие белая и красная сущности, вместе с
ветром\-умом. Когда они пребывают в нечистом состоянии, они образуют основу нашего
обычного тела, речи и ума. Это также тонкие факторы, которые затемняют три проявления
(snang gsum gyi sgrib), а также препятствуют совершенству истинного Ясного света, как есть
(ji bzhin). Только путь Ваджраяны считается противоядием для этих факторов. Вот почему
говорят, что единственный путь следования — это Тайная Мантра.
Когда осуществлена и полностью понята связь между основой очищения, процессом
очище\-ния и результатом очищения, достигают знания всех ключевых моментов пути
Ваджраяны.
\end{siderules}

\subsection{Три самадхи}
\vspace{0.2cm}
В самом начале практика Стадии Зарождения проходит через [этапы] трех самадхи:\\
\\
\vspace{0.2cm}
\begin{tabular}{ll}
1 & самадхи Таковости;\\
2 & самадхи Ясного Света;\\
3 & причинное самадхи.\\
\end{tabular}
\\
Далее описывается сущностная природа самадхи таковости (de bzhin nyid kyi ting ge 'dzin
gyi ngo bo): Внутри и вне себя, ум независит от какой либо основы. Нет корня, из которого
он вырастает. Он не существует в какой либо онтологической крайности (dngos ро' mtha'). Он
не мужского, не женского и не среднего рода. У него нет цвета, структуры или формы.
Однако, поскольку он по своей природе является Ясным Светом (rang bzhin gyi 'od gsal), он
также не является и простым ничто. В результате воспоминания Мудрости Дхарматы, как
она есть на самом деле (chos nyid ji ltar ba'i ye shes), существование смерти очищается в
Дхармакайю. Уверенность в том, что вещи постоянны, очищается также как и сфера без
формы (gzugs med khams). Таким образом, это называют "самадхи таковости".
Самадхи Ясного Света ('od gsal) называется так, поскольку естественное сияние этого
великого пустого Света является уравнивающим и объединяющим состраданием по
отношению ко всем живым существам, а также оно освобождает от нигилистических
воззрений и сферу форм (gzugs kyi khams). Кроме того, непостижимая проявляющаяся
энергия (rol rtsal) этой распространяющей Мудрости (mched ра'i ye shes) готовит почву для
превращения промежуточного состояния\-бардо в Самбхогакайю. Это так называемое "все
освещающее (kun tu snang ba) самадхи".\\
\\
Из этого само осознавание (rigра nyid) появляется затем в форме слогов А, ХУМ или
ХРИ. Это процесс известен как "причинный метод", а также "коренной метод". Причинное
самадхи очищает сознание нынешнего момента существования, которое готово войти в
новое место пребывание. Кроме того оно очищает сферу желаний ('dod khams) и приводит к
созреванию рождение в Нирманакае. Таково причинное самадхи.

\subsection{Зарождение поддерживающей \\ и поддерживаемой мандал}

Заложив основу для Стадии Зарождения этими тремя видами самадхи, далее можно
приступать к зарождению поддерживаю\-щего и поддерживаемого. Точная природа этого
процесса изложена в обширном собрании тантр традиции ранних переводов Ньингма, в связи
с триадой Зарождения, Заверше\-ния и Великого Завершения, в таком как "Великая тайная
сущность" (dpal gsang bа'i snying po). В этом подходе причинный или коренной метод
приводит к основополагаю\-щему способу визуализации дворца божества и трона. И это, в
своюо чередь, ведёт к невообразимому (bsamyas) способу медитации на всей
поддерживаемой мандале.\\
\\
Кроме того, медитатация на форме чудесного дворца в бескрайнем пространстве
благословляет (byin brlab) нечистую природу сосуда (т.е. этого мира) как Акаништху.
Тридцать семь факторов Просветления (byang chub kyi chos sum cu rtsa bdun) — это то, что
устанавливает сущностную связь (sbyor bа'i ngo bo) между основой, путем, Плодом,
результатом и чистотой. Природа этой связи объясняется в девятой главе текста
"Великая колесница (shing rta chen mо)".\\
\\
Здесь, однако, мы сосредоточимся в основном на стадиях визуализации поддерживаемого
божества (brten ра), связывая способ, которым возникает ваджрное тело, со Стадией
Зарож\-дения. Данный процесс очищает основу (объект) очищения (sbyang gzhi).
Втантре "Херука Галпо" (he ru kа gal ро) сказано:
\begin{verse}
Первое — Шуньята и Бодхичитта,\\
Второе — возникновение Семени,\\
Третье — совершенная форма\\
Четвёртое — установление [коренного] Слога.\\
\end{verse}
Здесь говорится о том, что смерть и промежуточное состояние очищаются Шуньятой
(Пустотностью) и Бодхичиттой. Сознание в форме бестелесного духа (driza), готовое войти
в соединение семени и яйцеклетки, очищается собиранием семени (sa bon bsdu ba). Затем
постепенно развивается тело, которое сгущается десятью ветрами (энергиями) (гlung bcu).
Этот процесс очищается совершенной формой (gzugs rdzogs ра). Будучи рождёнными,
чувственные способности (органы вос\-приятия) (dbang ро) проявляют активность (sad ра) по
отноше\-нию к своим объектам. И это, в свою очередь, очищается установлением слога (yi ge
'god ра) и т.д. Данное объяснение имеет отношение к символам четырёх манифестаций
просвет\-ления (mngon byang bzhi' brda).\\
\\
Однако здесь будет дано объяснение с точки зрения постепен\-ного развития, что
соответствует общему взгляду "Собрания Тантр (rgyud sde)". Согласно отцовским тантрам
зарождение происходит посредством ритуала трёх важдр, в то время как согласно
материнским тантрам это проис\-ходит посредством пяти мани\-фестаций просветления. В
обеих этих традициях, основа очищения связана с чистотой процесса очищения.\\
\\
\begin{siderules}
Второй раздело бращен к трем самадхи, подходу, который является особым для традиции
ранних переводов школы Нингма. В других системах ничего не происходит, кроме
зарождения промежуточного состояния (bar srid bskyed), когда уже было собрано накопление
Мудрости. Но в данной традиции говорится, что сущность (snying ро) пустотности — это
сострадание, и что импульс, происходящий из союза, приводит к возникновению чудес\-ной
эманации (sprul pa'i lha) с телом, ликом и руками, а также к осуществлению блага других
посредством четырех видов просветленной активности. Таков безошибочный путь Великой
колесницы, соединения двух накоплений (tshogs).\\
\\
В пределах нечистого сансарического существования невозможно войти во чрево, если
желаниеи привязанность (sred len) неподдерживают карму, которая подталкивает к
будущему рождению. И если эта карма поддерживается, тогда происходит зачатие. Таким же
образом, когда поддерживаются импульсом ('phen pas gsos btab) великого сострадания, тогда
зарождают мандалу божеств, которая возникает из состояния сияющей пустотности. Это
соответ\-ствует объекту очищения в рамках Сансары. И это также соответствует тому, что
происходит в рамках результата завершения этого процесса. В этот момент
неконцептуаль\-ное сострадание воплощается из Дхармакаи и работает на благо бесконечного
количества живых существ.\\
\\
Пока не теряют из виду эликсир (rtsi) пустотности и сострадания, Стадия Зарождения не
может привести к укреплению обычного состояния. Медитируя в соответ\-ствии с путем
окончательного совершенства (nges rdzogs) достигают устойчивости в самадхи таковости.
Благодаря этой устойчивости зарождение и завершение пребывают в союзе, а также в потоке
ума возникает подлинная Стадия Завершения, которую Буддаачарья Джнянапада описывает
как недвойственность глубины и ясности (zab gsal gnyis med). Джнянапада поддерживает
медитацию, в которой Стадия Зарождения запечатана Стадией Заверше\-ния. И хотя сказано,
что ранние учителя непонимали этого момента, данный взгляд соответствует особому
способу представления этой темы в ранних традициях школы Нингма.\\
\\
Когда медитируют на трех самадхи в соответствие с путем окончательного совершенства, те,
кто получил посвящение и хранит обеты самаи, в начале должен быть введен (ngo sprod) в
истинную природу воззрения. Следующая ступень — это обретение уверенности (yid ches) в
этом воззрени и посредством опоры на логические рассуждения (gtan tshig). Затем
необходимо поместить безмятежный и естественный ум в состоянии безыскусной простоты
(mа bcos spros bral) посредством одного из двух методов: либо посредством пребывания
после прозрения (mthong ba'i rjes la 'jog pa), либо посредством пребывания в
непосредственности полного осознавания (rigра spyi blug su 'jog pa). Это позволяет пресечь
поток перемещающихся воспоминаний и мыслей, а также освободиться от ошибок вялости и
смятения (bying rgod). Знакомство с этим состоянием вводит ветры ума в центральный канал.
Видимые проявления десяти знаков (rtags bcu) достигают состояния совершенства и, в
лучшем случае, могут даже зарождать символическую Мудрость, которая опирается на три
проявления.\\
\\
И даже если это не так, можно тренировать ум, упрочивая понимание, что пустотность
свободна от концептуальных измышлений (spros ра). Затем, когда не теряют из виду это
понимание, то видят, что Сансара проявляется, но не существует. Тогда появляется
возможность развивать подобные иллюзии и свободное от фиксации сострадание, которое
пронизывает все пространство. Благодаря знаком\-ству с этим, когда ум фиксируется (blо
bzhag) на пустотности, без усилий возникает чувство сострадания ко всем суще\-ствам,
которые не постигают этого и которые обманываются на самом деле несуществующими
проявлениями. И даже когда медитируют на сострадании, не будет сделана ошибка
конкретизации (а 'thas) себя и других. Вместо этого появляется уверенность в пустотности, в
том факте, что проявления возникают во взаимной зависимости и иллюзорны, а также в том,
что их истинная природа не установлена.\\
\\
Знакомство (nges shes) с пустотностью, которая имеет своей сущностью сострадание — это
ключевой момент первых двух самадхи — Таковости и Ясного света. По этой причине
отсутствие одного из них не позволит вам достичь Совершенного Просветления (rdzogs pa'i
byang chub). Поэтому не стоит даже говорить о том, что будет, если не будет обоих. Как
писал Сараха:
\begin{verse}
Без сострадания воззрение пустотности —\\
Нет достижения высшего Пути.\\
Если же вы медитируете только на сострадании,\\
Как вы освободитесь от Сансары?\\
\end{verse}
Эти два фактора являются тем, что превращает Стадию Зарождения в путь Великой
колесницы. И наоборот, любая Стадия Зарождения без этих двух [самадхи] ничем не
отличается от не-Буддийской [практики].\\
\\
Эта пустотность с состраданием в качестве сущности (snying ро) проявляется как А, ХУМ,
ХРИ и другие причинные семенные слоги (rgyu'i sa bon). Когда вы медитируете на этих
слогах и знакомитесь с ними, то можно пребывать в медитативном состоянии (mnyam par
bzhag) так долго как хочется. И когда могут это делать, то могут эманировать бесчисленное
количество слогов, так, что они заполняют все пространство. Затем снова собирают их в
первичном слоге и погружаются в медитативное состояние. Необходимо продолжать
упражняться в этих практиках, пока не будут достигнуты восемь измерений ясности и
устойчивости (gsal brtan gyi tshad brgyad). Этот процесс известен как "тренировка тонких
слогов (phra ba yig 'bru la bslab ра)". Знакомство только с этимпроцессом позволяет достичь
всех духовных достижений (dngo sgrub) и знаков реализации (grub rtags).
Если нет возможности медитировать на трех самадхи в соответствие с Путем определенного
совершенства (lam nges rdzogs), то можно вместо этого использовать подход
медитации устремления (mos sgom). В таком случае в начале каждой практики должны быть
осуществлены различные аспекты ритуала. Не довольствуйтесь простым произнесением
слов! Вместо этого расслабьтесь изнутри и пребывайте в медитативном состоянии до
реализации самадхи Таковостии Ясного света. Затем постепенно медитируйте на причинном
самадхи и других стадиях практики. Пока практика Стадии Зарождения не укоренится в
первично мэлементе практики (nyams len gyi gtso bo) — пустотности и сострадании — не
следует отдавать приоритет рецитации, испусканию и втягиванию (bzlas ра dang spro bsdus),
а также другим подобным факторам.\\
\\
Во время практики Стадии Зарождения, медитируете ли вы и спользуя подход медитации
устремления или определенного совершенства, необходи момедитировать на трех самадхи,
неотдельных друг от друга. Но не\-достаточно практиковать их время от времени. Например,
когда на стене рисуют изображение, то стена служит основой для рисования, поскольку, если
ее нет, тогда не на чем рисовать. И даже если стена есть, но она не от штукатурена, на ней
также нельзя ничего нарисовать. Штукатурка в данном случае является содействующим
условием (lhan cig byed pa'i rkyen) изображения. Когдаже изображение нарисовано, стена,
штукатурк а и само изображе\-ние по-прежнему остаются. Таким же образом, даже когда
медитируют на всём круге мандалы, который возникает из причинного самадхи, в качестве
божества проявляется именно пустотность и сострадание. Поэтому эти три самадхи должны
практиковаться в единстве.
\paragraph{}
Но если обычные проявления еще не очистили до пустоты, то как вы можете медитировать
на круге мандалы? Тот факт, что все явления являются пустотностью, и что Сансара и
Нирвана нераздельны, есть причина, по которой мы можем это актуализировать благодаря
медитации на круге мандалы. Иными словами, пустотность — это основа для Стадии Зарождения.
Как сказано:
\begin{verse}
Для кого пустотность возможна,\\
Для того все возможно.\\
\end{verse}
Если бы все явления не были пустыми, а обычные проявления были истинно установлены
(bden par grub), тогда медитация Стадии Зарождения была бы невозможна. И, как
указывается в следующей цитате:
\begin{verse}
Можно посвятить пшено в рис,
Но рис все равно не появится.
\end{verse}
Даже если все явления таким образом постигаются как пустые, без импульса великого
сострадания вы не сможете воплотить рупакаи, чтобы помочь другим. Это подобно
шравакам и пратьекабуддам, которые входят в состояние прекращения ('gags ра) и не
помогают другим эманациями рупакаи.\\
\\
Когда понимают этот момент, то это подобно следующему высказыванию: "Все эти явления
подобны иллюзии, а рождение подобно прогулке по парку..." Иначе говоря, больше не
пребывают в существовании, тогда как вследствие сострадания нецепляются за покой (zhi).
Таков великий Путь сыновей Победоносного. По этим причинам, понимание того, что три
самадхи неизолированы друг от друга, это важнейший момент.\\
\\
Сегодня большинство из тех, кто имеет некоторое переживание Стадии Зарождения,
сохраняют злобу на своем смертном одре. В итоге они перерождаются в сфере Владыки
Смерти (gshin rje). И это отклонение приводит к тому, что они становятся демонамии
приносят вред живым существам. Причина этого состоит в том, что эти люди не освоили (mа
goms ра) пустотность с состраданием в качестве своей сущности. Есть множество людей,
обладающих интеллектуальным пониманием пустотности, могущих ясно визуализировать на
Стадии Зарождения, а также полностью завершивших повторение мантры, но которые тем не
менее заканчивают тем, что перерождаются как демоны. С другой стороны, вы никогда не
увидите злобного вурдалака, который быо бладал эликсиром великого сострадания.
\end{siderules}

\subsubsection{Ритуал трех ваджр}

Символические атрибуты, такие как пятиконечная ваджра, очищают ум. Ясно медитируя
на их форме, основа (объект) очищения — ум, очищается [и превращается] в ваджрный ум.
И подобным же образом, основа очищения — обычная речь, совершенствуется в ваджрную речь.
Процесс очищения в этом случае состоит в превращении этих символических предметов
(атрибутов) (phyag mtshan) или украшение (mtshan ра) их в медитации слогом ХУМ и
другими. Следующая стадия очищения предполагает ритуал испускания и поглощения (spro
bsdus) лучей света, который приносит двойную пользу, а также последующее превращение в
форму божества, наделенную всеми украшениями и одеяниями. Эта медитация развивает
тело, являющееся здесь объектом очищения, до ваджрного тела. Данный этап образует
ритуал трех ваджр.\\
\\
И эти три [ваджры] также могут быть применимы к процессу рождения из чрева. Первая
стадия очищает слияние красного и белого элементов, а также последующее вхождение
сознания в промежуточное состояние. Вторая стадия очищает пятичастный процесс развития
зародыша в чреве, состоящий из [формирования] овальной формы и остальных стадий,
которые следуют за схождением семени, крови (яйца) и сознания. Третья стадия очищает
рождение, которое возникает, когда соединяются прежде рассеянные элементы и полностью
формируются тело и органы чувств.\\
\\
Этот процесс также образует стадию того, что происходит послео бретения Плода.
Взаимозависимость (rten 'brеl) создается определенными действиями, которые
манифестируются татхагатами. Особенно это относится к просветленной активности
вхождения в чрево ваджрной царицы, рождения и т.д., которая демонстрируется Буддами,
когда они обретают форму нирманакаи, чтобы покорять существ необходимыми методами.

\subsubsection{Пять актуализаций Просветления}

В этом обсуждении мы объединяем пять внешних актуализаций Просветления (phyi' mngon byang)
йога-тантры (rnal 'bуог rgyud) с пятью внутренними актуализациями
Просветления, описываемые в традиции маха-йоги (rnal 'bуог chen ро), соединяя причину со
способом достижения Плода. В тантре "Пространства ясности (klong gsal)" сказано:
\begin{verse}
Зачатие связано с пятью аспектами Просветления,\\
Десять месяцев соспособом развертывания десятиу ровней,\\
И рождение — с естественной нирманакаей;\\
Так, естественным образом, воплощаются три Тела.
\end{verse}
Как здесь сказано, период, начинающийся с промежуточного состояния и
продолжающийся до момента, когда ищут физическое тело, связан с путем накопления
(tshogs lam), тогда как действительное обретение физической формы в чреве связано с путем
соединения (sbyor lam). Период с момента зачатия и далее отмечается как путь медитации
(sgomgyi lam). В сущности, десять месяцев представляют собой десять уровней (sa bcu),
которые ведут к осуществлению естественной нирмана\-каи, нераздельного состояния без
тренировки (mi slob pa'i zung 'jug rang bzhin sprul pa'i sku).
Здесь мы связываем определенный процесс очищения, совершенствования и
созре\-вания с различными стадиями четырех видов рождения.\\
\\
1) Первая манифестация просветления (mngon byang) — этолунный диск. Она
очищает и вымывает следующие объекты очищения: совокупность формы (gzugs kyi phung
ро), элемент пространства (nаm mkha'i khams), тело и сознание — основу всего (lus kun gzhi'i
rnam shes), омрачение-клешу неведения (nyon mongs ра gti mug), мужской элемент (семя,
pha'i khams), связанный с рождением из чрева, влагу, связанную с рождением из тепла и
влаги, мужское (pha), связанное с рождением из яйца, аспект пустоты (stong pa'i cha),
связанный с чудесным рождением и т.д. В контексте Стадии Завершения (rdzogs rim)
нисходят шестнадцать радостей (dga' ba) и первая Мудрость различающего постижения (so
sor rtogs ра) проявляется как природа искусных средств (thabs kyi rang bzhin). Иными
словами, ум совершенствуется в своей сущности (ngo bo). Что касается аспекта Пути, то
медитируют, что это превращается в лунный диск. Что же касается Плода, то это созревает в
виде тридцатидвух благих знаков полного Просветления и актуализации Мудрости
отражения (mеlong ye shes).\\
\\
2) Вторая манифестация Просветления — это солнечный диск. Он очищает и вымывает
следующие объекты очищения: совокупности ощущения (tshor ba), элемент земли (sa),
омрачение-клешу сознания(?), жадность и гордость, красный женский элемент, связанный с
рождением из чрева, женское, связанное с рождением из яйца, тепло, связанное с рождением
из тепла и влаги, аспект ясности (gsal ba), связанный с чудесным рождением и т.д. В
процессе Стадии Завершения яйцо и семя полностью очищаются, становясь по своей сути
одновременным возникновением блаженства и пустоты (bde stong lhan cig skyes). Это
предполагает процесс восходящей опоры (mas brtan) — природы мудрости. Двадцать
элементов (дхату) (т.е. форма и остальные скандхи в сочетании с четырьмя элементами)
возникают как двадцать пустотностей (Мудрость базового пространства явлений
(дхармадхату) и остальные Мудрости в сочетании с четырьмя неизмеримыми) и являются
совершенными как Пробужденный ум. Это вторая Мудрость. Что касается аспекта Пути, то
медитируют, что это превращается в солнечный диск. И это, в свою очередь, приводит к
созреванию восьмидесяти чудесных второстепенных признаков, которые возникают в
состоянии Плода, а также к полному Просветлению и актуализации Мудрости равенства
(mnyam nyid kyi ye shes).\\
\\
3) Третья манифестация Просветления — это семенной слог и атрибут руки (sa bon
phyag mtshan). Она очищает и вымывает следующее: вхождение в промежуточное состояние
сознания (bardo) (бестелесный длящийся ум) между семенем и яйцом, совокупность
восприятия ('du shes), элемент огня (mе), речь (ngag), ментальное сознание(yid kyi rnam shes),
омрачение-клешу желания ('dod chags), а также промежуточное состояние сознания бардо,
которое входит в любой из четырех видов рождения. Что касается Стадии Завершения, то
семя и яйцо очищаются обратно-вспять (ldog tu dag ра), после чего кармические ветры (las
rlung) и связь времени (dus sbyor) прекращаются. Так достигают совершенства третьей
Мудрости ваджрной устойчивости (bstan pa'i rdo rje). Что касается а спекта пути, то
медитируют на том, что это превращается в семенной слог и атрибут руки. Что же касается
Плода, то это созревает в природу постижения отчетливости (mа 'dres ра) всех явлений, а
также приводит к полному Просветлению Мудрости различения (so sor rtog  pa'i ye shes).\\
\\
4) Четвертая манифестация Просветления — это смешение семенного слога и
атрибута руки в один вкус (гоg cig tu 'dres ра). Это очищает и вымывает следующие факторы:
смешение семени, яйца и сознания в один вкус, совокупность формирующих факторов ('du
byed), элемента воздуха (rlung), пять чувственных сознаний, омрачение-клешу зависти (phrag
dog), а также смешение сознания промежуточного состояния, семени и яйца при рождении
из чрева и из яйца, смешение сознания, тепла и влаги при рождении из тепла и влаги, а также
смешение лишь пустой ясности (gsal la stong ра) чудесно горождения с сознанием
промежуточного состояния. На Стадии Завершения функционирование семени, яйца и
ветров полностью очищается. Сущность Держателя причинной ваджры (rdo rje 'dzin ра)
совершенствуется, поскольку это четвертая Мудрость знания всего многообразия (snyed
ра), имеющая ваджрную природу (bdag nyid) и охватывающая все познаваемое. Что касается
аспекта пути, то визуализируют, что сущность этого смешивания превращается в
светящуюся сферу тигле ('od kyi thig le). Когда достигают Плода, это созревает и
воплощается в виде активностей всех Будд, которая сливается в едином вкусе и приводит к
совершенному Просветлению Мудрости осуществления всего (bya ba grub pa'i ye shes).\\
\\
5) Пятая манифестация Просветления — это совершенная форма, которая
происходит от этого процесса. Она очищает и вымывает следующие факторы: завершение
периода беремен\-ности, совокупность сознания, элементводы (chu), аспект цепляния за
Реальность восьми собраний (chos nyid kyi tshogs brgyad), ментальное сознание, омрачение —
клешу гнева (zhes dang), а также состояние, в котором телесные органы чувств,
соответствующие одному из четырех способов рождения, полностью развиваются. Это
очищается и становится естествен\-ной нирманакаей. Способ, которым это развивается и
созревает на Стадии Завершения, следующий: Семя, яйцо и тонкие прана—ум становятся
своей совершенно чистой сущностью (rnam par dag pa'i ngo bo). К тому же, результирую\-щий
держатель ваджры и пятая Мудрость (?), воплощение Татхагат, становятся нашей сущностью
(bdag nyid). Последую\-щее относится к результирующему состоянию полного освобож\-дения
(rnam par grol ba), естественной Мудрости Будды, которая постигает явления такими, каковы
они на самом деле (ji lta ba). На этом пути медитируют на совершенной форме божества, а
также на его украшениях и одеждах. Когда достигают Плода, то созревает нераздельная
природа Мудрости-пространства (ye shes dbyings), которая свободна от любых препятствий,
что ведет к полному Просветлению Мудрости Дхармадхату (dbying skyi ye shes). В тантре
сказано:
\begin{verse}
Луна — это Зерцалоподобная Мудрость,\\
Семь из семи — это Равенство.\\
Различающее постижение описывается\\
Как семенной слог и атрибуты рук божества.\\
Становление всего одним — это само Усилие,\\
А совершенство — Дхармадхату.
\end{verse}
Хотя есть четыре способа медитации Стадии Зарождения, когда каждый связан с
четырьмя видами рождения, их необходимо использовать таким образом, чтобы они
соответ\-ствовали индивидуальным предрасположенностям (rang rang gi bag chags) и степени
знакомства с практикой. Владыка Победоносных, Лонгченпа, писал:\\
\begin{verse}
\smallsize
Хотя есть четыре способа медитации, ты должен использовать \\
Тот, который направлен на преобладающий способ рождения. \\
Чтобы очистить предрасположенности, медитируй используя все.\\
В частности, сначала медитируй соответственно рождению из яйца,\\
А когда появится устойчивость — соответственно рождению из чрева.\\
При большей устойчивос тимедитируй соответственно рождению из тепла и влаги,\\
Когда же полностью освоишься, используй эту истинную устойчивость,\\
Чтобы мгновенно визуализировать, соответственно чудесному рождению.
\normalsize
\end{verse}
