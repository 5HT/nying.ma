\addtocontents{toc}{\protect\newpage}
\section{Гуру Йога}
\\
\subsection{Визуализация}
\\
\ti
ཨེ་མ་ཧོཿ \\
རང་སྣང་ལྷུན་གྲུབ་དག་པ་རབ་འབྱམས་ཞིང་༔ \\
བཀོད་པ་རབ་རྫོགས་ཟང་མདོག་དཔལ་རིའི་དབུས༔\\
རང་ཉིད་གཞི་ལུས་རྡོ་རྗེ་རྣལ་འབྱོར་མ༔\\
ཞལ་གཅིག་ཕྱག་གཉིས་དམར་གསལ་གྲི་ཐོད་འཛིན༔\\
ཞབས་གཉིས་དོར་སྟབས་སྤྱན་གསུམ་ནམ་མཁར་གཟིགས༔\\
སྤྱི་བོར་པདྨ་འབུམ་བརྡལ་ཉི་ཟླའི་སྟེང་༔\\
སྐྱབས་གནས་ཀུན་འདུས་རྩ་བའི་བླ་མ་དང་༔\\
དབྱེར་མེད་མཚོ་སྐྱེས་རྡོ་རྗེ་སྤྲུལ་པའི་སྐུ༔\\
དཀར་དམར་མདངས་ལྡན་གཞོན་ནུའི་ཤ་ཚུགས་ཅན༔\\
ཕོད་ཁ་ཆོས་གོས་ཟ་བེར་འདུང་མ་གསོལ༔\\
ཞལ་གཅིག་ཕྱག་གཉིས་རྒྱལ་པོ་རོལ་པའི་སྟབས༔\\
ཕྱག་གཡས་རྡོ་རྗེ་གཡོན་པས་ཐོད་བུམ་བསྣམས༔\\
དབུ་ལ་འདབ་ལྡན་པདྨའི་མཉེན་ཞུ་གསོལ༔\\
མཆན་ཁུང་གཡོན་ན་བདེ་སྟོང་ཡུམ་མཆོག་མ༔\\
སྦས་པའི་ཚུལ་གྱི་ཁ་ཊྭཱཾ་རྩེ་གསུམ་བསྣམས༔\\
འཇའ་ཟེར་ཐིག་ལེ་འོད་ཕུང་ཀློང་ན་བཞུགས༔\\
ཕྱི་འཁོར་འོད་ལྔའི་དྲྭ་བས་མཛེས་པའི་ཀློང་༔\\
སྤྲུལ་པའི་རྗེ་འབངས་ཉི་ཤུ་རྩ་ལྔ་དང་༔\\
རྒྱ་བོད་པཎ་གྲུབ་རིག་འཛིན་ཡི་དམ་ལྷ༔\\
མཁའ་འགྲོ་ཆོས་སྐྱོང་དམ་ཅན་སྤྲིན་ལྟར་གཏིབས༔\\
གསལ་སྟོང་མཉམ་གནས་ཆེན་པོའི་ངང་དུ་གསལ༔\\
\ru
\newpage
ОМ А ХУМ\\
Самовозникшая спонтанная полностью чистая Сфера,\\
Совершенно расположенная Славная Медноцветная Гора, \\
В центре я в виде Ваджрайогини, с одним лицом, двумя руками,\\
Красная, сияющая, держит дигуг и капалу,\\
Две ноги в заигрывающей позе, три глаза глядят в пространство.\\
Над моей головой на лунном и солнечном дисках \\
поверх распустившегося тысячелепесткового лотоса\\
Коренной Учитель, воплощение всех объектов прибежища,\\
в форме Нирманакаи Падмасамбхавы.\\
В расцвете юности, с легким румянцем,\\
в парчовом одеянии, халате цвета неба, \\
монашеской накидке и царской мантии\\
с оним лицом, двумя руками, в игривой царско позе.\\
В правой руке ваджра, в левой череп с сосудом.\\
На голове пятилепестковая лотосовая шапка.\\
Прижимая згибом левой руки \\
высшую супругу блаженства-пустоты,\\
сокрытую в виде тризубца,\\
восседает в сиянии радужных лучей и сфер света.\\
В окружении совершенного сплетения пятицветного света\\
двадцать пять проявлений, царь и подданые,\\
ученые и мастера, держатели ведения Индии и Тибета, \\
божества, летящие в пространстве, \\
охранители учения и держатели обетов собраны подобно облаку.\\
Все — великая равность светоносносности и пустоты.
\newpage
\subsection{Семистрочная молитва}
\\
\ti
ཧཱུྂ༔ ཨོ་རྒྱན་ཡུལ་གྱི་ནུབ་བྱང་མཚམས༔\\
པདྨ་གེ་སར་སྡོང་པོ་ལ༔\\
ཡ་མཚན་མཆོག་གི་དངོས་གྲུབ་བརྙེས༔\\
པདྨ་འབྱུང་གནས་ཞེས་སུ་གྲགས༔\\
འཁོར་དུ་མཁའ་འགྲོ་མང་པོས་བསྐོར༔\\
ཁྱེད་ཀྱི་རྗེས་སུ་བདག་བསྒྲུབ་ཀྱི༔\\
བྱིན་གྱིས་བརླབ་ཕྱིར་གཤེགས་སུ་གསོལ༔\\
གུ་རུ་པདྨ་སིདྡྷི་ཧཱུྂ༔\\
\\
\ru
На северо-западной границе священной страны Удиянны,\\
Посреди озера Данакоша, в ложе цветка на стебле лотоса,\\
Ты обрел чудесные высшие достижения,\\
Прославленный как Лотосорожденный,\\
Окружен свитой из множества дакинь и видьядхар.\\
Я буду практиковать следуя тебе!\\
Молю приди и благослови!
\newpage
\subsection{Семь ветвей}
\\
\ti
ཧྲཱི༔ བདག་ལུས་ཞིང་གི་རྡུལ་སྙེད་དུ༔\\
རྣམ་པར་འཕྲུལ་པས་ཕྱག་འཚལ་ལོ༔\\
དངོས་བཤམས་ཡིད་སྤྲུལ་ཏིང་འཛིན་མཐུས༔\\
སྣང་སྲིད་མཆོད་པའི་ཕྱག་རྒྱར་འབུལ༔\\
སྒོ་གསུམ་མི་དགེའི་ལས་རྣམས་ཀུན༔\\
འོད་གསལ་ཆོས་སྐུའི་ངང་དུ་བཤགས༔\\
བདེན་པ་གཉིས་ཀྱིས་བསྡུས་པ་ཡི༔\\
དགེ་ཚོགས་ཀུན་ལ་རྗེས་ཡི་རང་༔\\
རིགས་ཅན་གསུམ་གྱི་གདུལ་བྱ་ལ།\\
ཐེག་གསུམ་ཆོས་འཁོར་བསྐོར་བར་བསྐུལ༔\\
ཇི་སྲིད་འཁོར་བ་མ་སྟོངས་བར༔ \\
མྱ་ངན་མི་འདའ་བཞུགས་གསོལ་འདེབས༔ \\
དུས་གསུམ་བསགས་པའི་དགེ་རྩ་ཀུན༔ \\
བྱང་ཆུབ་ཆེན་པོའི་རྒྱུ་རུ་བསྔོ༔ \\
\\
\ru
Совершаю простирания, преобразив свое тело
во столько тел, сколько частиц во вселенной.\\
Всеозможные материальные подношения и те, что созданы в уме силой медитации;
подношу всю вселенную в одно мгновение.\\
Всю свою неблагую карму, созданную телом, речью и умом
очищаю в ясном свете Дхармакаи.\\
Как относительным так и абсолютным,
сорадуюсь всем накопленым заслугам и мудрости.\\
Для блага существ с тремя видами способностей
прошу вас  поворачивать колесо Дхармы трех колесниц.\\
Пока сансара полностью не опустеет и существа не обретут свободу,
молю, пожалуйста, оставайтесь среди нас и не входите в паринирвану.\\
Все заслуги и благую карму прошлого, настоящего и будущего
посвящаю всем существам, чтобы они стали Буддами.
\\
\subsection{Созревание Достижений}
\\
\ti
༈ རྗེ་བཙུན་གུ་རུ་རིན་པོ་ཆེ༔ \\
ཁྱེད་ནི་སངས་རྒྱས་ཐམས་ཅད་ཀྱི༔ \\
ཐུགས་རྗེ་བྱིན་རླབས་འདུས་པའི་དཔལ༔ \\
སེམས་ཅན་ཡོངས་ཀྱི་མགོན་གཅིག་པུ༔ \\
ལུས་དང་ལོངས་སྤྱོད་བློ་སྙིང་བྲང་༔ \\
ལྟོས་པ་མེད་པར་ཁྱེད་ལ་འབུལ༔ \\
འདི་ནས་བྱང་ཆུབ་མ་ཐོབ་བར༔\\
སྐྱིད་སྡུག་ལེགས་ཉེས་མཐོ་དམན་ཀུན༔\\
རྗེ་བཙུན་ཆེན་པོ་པད་འབྱུང་མཁྱེན༔\\
ཨོཾ་ཨཱཿཧཱུྂ་བཛྲ་གུ་རུ་པདྨ་སིདྡྷི་ཧཱུྂ༔\\
\\
\ru
Благородный владыка, Непревзойденная драгоценность,\\
Ты — славное воплощение милосердия \\
и благословения всех Будд,\\
Кдинственный защитник всех существ,\\
Тело, богатство, сердце и мышление, подношу тебе без сомнений!\\
С этого мгновения, пока не обрету пробуждение,\\
В счастье и страдании, в радости и печали, в успехе и неудаче,\\
Всевеликий благородный владыка, помни меня!
\newpage
\subsection{Благопожелания}
\\
\ti
༈ བདག་ལ་རེ་ས་གཞན་ན་མེད༔\\
ད་ལྟའི་དུས་ངན་སྙིགས་མའི་འགྲོ༔\\
མི་བཟད་སྡུག་བསྔལ་འདམ་དུ་བྱིངས༔\\
འདི་ལས་སྐྱོབས་ཤིག་མ་ཧཱ་གུ་རུ༔\\
དབང་བཞི་བསྐུར་ཅིག་བྱིན་རླབས་ཅན༔\\
རྟོགས་པ་སྤོར་ཅིག་ཐུགས་རྗེ་ཅན༔\\
སྒྲིབ་གཉིས་སྦྱོངས་ཤིག་ནུས་མཐུ་ཅན༔\\
ཨོཾ་ཨཱཿཧཱུྂ་བཛྲ་གུ་རུ་པདྨ་སིདྡྷི་ཧཱུྂ༔\\
\\
\ru
У меня нет иной опоры кроме тебя.\\
Существа нынешнего времен упадка\\
Тонут в пучине нестерпимых страданий.\\
О, Великий Гуру, защити нас от них!\\
Благословенный, даруй четыре посвящения.\\
Милосердный учитель, усиль реализацию.\\
Могущественный, очисти два омрачения.\\
\newpage
\subsection{Растворение}
\\
\ti
༈ ནམ་ཞིག་ཚེ་ཡི་དུས་བྱས་ཚེ༔\\
རང་སྣང་རྔ་ཡབ་དཔལ་རིའི་ཞིང་༔\\
ཟུང་འཇུག་སྤྲུལ་པའི་ཞིང་ཁམས་སུ༔\\
གཞི་ལུས་རྡོ་རྗེ་རྣལ་འབྱོར་མ༔\\
གསལ་འཚེར་འོད་ཀྱི་གོང་བུ་རུ༔\\
གྱུར་ནས་རྗེ་བཙུན་པད་འབྱུང་དང་༔\\
དབྱེར་མེད་ཆེན་པོར་སངས་རྒྱས་ཏེ༔\\
བདེ་དང་སྟོང་པའི་ཆོ་འཕྲུལ་གྱི༔\\
ཡེ་ཤེས་ཆེན་པོའི་རོལ་པ་ལས༔\\
ཁམས་གསུམ་སེམས་ཅན་མ་ལུས་པ༔\\
འདྲེན་པའི་དེད་དཔོན་དམ་པ་རུ༔\\
རྗེ་བཙུན་པདྨས་དབུགས་དབྱུང་གསོལ༔\\
གསོལ་བ་སྙིང་གི་དཀྱིལ་ནས་འདེབས༔\\
ཁ་ཙམ་ཚིག་ཙམ་མ་ཡིན་ནོ༔\\
བྱིན་རླབས་ཐུགས་ཀྱི་ཀློང་ནས་སྩོལ༔\\
བསམ་དོན་འགྲུབ་པར་མཛད་དུ་གསོལ༔\\
ཨོཾ་ཨཱཿཧཱུྂ་བཛྲ་གུ་རུ་པདྨ་སིདྡྷི་ཧཱུྂ༔\\
\ru


