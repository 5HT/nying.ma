\small
\ti
༄༅། གདོད་ནས་མངནོ་སུམ་སངས་རྒྱས་ཀྱང་གང་ལ་གང་འདུལ་གཟུགས་སྐུ་འགགས་པ་མེད།\\
སྣ་ཚོགས་སྒྱུ་འཕྲུལ་ངོམ་ཡང་ཕུང་ཁམས་སྐྱེ་མཆེད་གཟུགས་དང་འཛིན་པ་བྲལ།\\
མི་ཡི་གཟུགས་སུ་སྣང་ཡང་མཁྱེན་བརྩེའི་འོད་ཟེར་སྟོང་འབར་རྒྱལ་བ་དངོས།\\
ཚེ་འདི་ཙམ་དུ་མ་ཡིན་གཏན་གྱི་སྐྱབས་སུ་ཁྱེད་བསྟེན་བྱིན་གྱིས་རློབས།\\
\\
\\
\ru
Хотя просветлен с самого начала, ты не перестаешь принимать
формы для приручения каждого в его личный способ.
Хотя виден как чудесным образом возникшие феномены,
ты на самом деле свобобден от скандх, элементов, чувств и цепляний.
Хотя ты появился в человеческом теле, в действительности ты Будда,
светящийся тысячами лучей всеведения и счастья.
Не только в этой жизни, но всегда, я принимаю в тебе прибежище,
Кхьенце Озер, наполни меня своими благословениями!\\
\\
\vspace{0.5cm}
\\
\scriptsize
\ti དེ་ལ་འདིར་རྫོགས་པ་ཆེན་པོ་ཀློང་ཆེན་སྙིང་ཐིག་གི་སྔོན་དུ་འགྲོ་བའི་ངག་འདོན་ཁྲིགས་སུ་བསྡེབས་པ་ལ།\\
\\
\ru Здесь подготовлена практика Дзогчен Лонгчен Ньингтик Нендро.\\


\normalsize
\newpage
\section*{Благословение речи}
\\
\ti
ཨོཾ་ཨཱཿཧཱུྂ།\\
ལྕེ་དབང་རྂ་ཡིག་ལས་བྱུང་མེས་བསྲེགས་ནས། \\
འོད་དམར་རྣམ་པའི་རྡོ་རྗེ་རྩེ་གསུམ་སྦུབས།\\
ཨཱ་ལི་ཀཱ་ལིའི་མཐའ་སྐོར་རྟེན་འབྲེལ་སྙིང་།\\
མུ་ཏིག་ཕྲེང་བ་ལྟ་བུའི་ཡིག་འབྲུ་ལས།\\
འོད་འཕྲོས་རྒྱལ་བ་སྲས་བཅས་མཆོད་པས་མཉེས།\\
སླར་འདུས་ངག་སྒྲིབ་དག་ནས་གསུང་རྡོ་རྗེའི།\\
བྱིན་རླབས་དངོས་གྲུབ་ཐམས་ཅད་ཐོབ་པར་འགྱུར།\\
\\
\ru
OM A ХУМ \\
Огнем, возникающим из слога Рам сжигается язык,\\
Возникает трехконечная ваджра из красного света,\\
Санскритские гласные и согласные, окружены мантрой\\
Сущности Взаимозависимого Происхождения,\\
Из слогов подобных жемчужному ожерелью,\\
Распространяется свет, свершающий радующие \\
подношения Буддам и Бодхисаттвам,\\
Собирается обратно, очищает омрачения речи, обретаются\\
Все благословения и сиддхи ваджрной речи.
\newpage
\subsection*{Мантра Гласных}
\\
\ti
ཨ་ཨཱ། ཨི་ཨཱི། ཨུ་ཨཱུ། རྀ་རཱྀ། ལྀ་ལཱྀ། ཨེ་ཨཻ། ཨོ་ཨཽ། ཨཾ་ཨཿ\\
\\
\ru
А аа И ии У уу Ри рии Ли лии Э ээ О оо Ам А \hspace{1cm} 7 раз\\
\\
\subsection*{Мантра Согласных}
\\
\ti
ཀ་ཁ་ག་གྷ་ང་། \\
ཙ་ཚ་ཛ་ཛྷ་ཉ། \\
ཊ་ཋ་ཌ་ཌྷ་ཎ། \\
ཏ་ཐ་ད་དྷ་ན། \\
པ་ཕ་བ་བྷ་མ། \\
ཡ་ར་ལ་ཝ། \\
ཤ་ཥ་ས་ཧ་ཀྵཿ \\
\\
\ru
КА  КХА  ГА ГХА НА\\
ЦА  ЦХА  ДЗА ДЗХА НЬЯ\\
ТРА ТХРА ДРА ДХРА НРА\\
ТА  ТХА  ДА ДХА НА\\
ПА  ПХА  БА БХА МА\\
Я   РА   ЛА ВА\\
ЩА  КХА  СА ХА КЩА \hspace{1cm} 7 раз\\
\newpage
\subsection*{Мантра Взаимозависимого Происходжения}
\\
\ti
ཨོཾ་ཡེ་དྷརྨཱ་ཧེ་ཏུ་པྲ་བྷཱ་ཝཱ་ཧེ་ཏུནྟེ་ཥཱནྟ་ཐཱ་ག་ཏོ་ཧྱ་ཝ་དཏ།\\
 ཏེ་ཥཱཉྩ་ཡོ་ནི་རོ་དྷ་ཨེ་ཝྃ་ཝཱ་དཱི་མ་ཧཱ་ཤྲ་མ་ཎཿསྭཱ་ཧཱ།\\
 \\
\ru
ОМ  ЙЕДХАРМА  ХЕТУПРАБХАВА\\
ХЕТУНТЕКХАН ТАТХАГАТО\\
ХАЙОВАДАТ ТЕКХАНЦАЙО НИРОДХА\\
ЭВАМВАДИ МАХАШРАМАНА! СОХА \hspace{1cm} 7 раз\\

\subsection*{Мантры Усиления Заслуг}
\\
\ti
ཨོཾ་སམྦྷ་ར་སམྦྷ་ར་བི་མ་ནཱ་སཱ་ར་མ་ཧཱ་ཛམྦྷ་ཧཱུྂ།\\
ཨོཾ་སྨ་ར་སྨ་ར་བི་མཱ་ན་སྐ་ར་མ་ཧཱ་ཛ་བ་ཧཱུྂ།\\
\\
\ru
ОМ САМБХАРА САМБХАРА\\
ВИМАНА САРА МАХА ДЗАМБХА ХУМ\\
ОМ МАРА МАРА ВИМАНА\\
КАРА МАХА ДЗАБА ХУМ \hspace{1cm} 7 раз\\
\\
\ti
ཨོཾ་རུ་ཙི་ར་མ་ཎི་པྲ་བརྡྷ་ནཱ་ཡེ་སྭཱ་ཧཱ།\\
ཧྲཱིཿབཛྲ་ཛི་ཧཱ་མནྟྲ་དྷ་ར་བརྡྷ་ནི་ཨོཾ།\\
\\
\ru
ОМ РУЦИ РАМАНИ ПРАВАРДАЕ СОХА\\
ХРИ БАДЗРА ДЗИХА\\
МАНТРА ДХАРА ВАРДХАНИ ОМ \hspace{1cm} 3 раза\\
\\
\scriptsize
\ti ཡི་དམ་གྱི་བཟླས་པའི་ཐོག་མར་འདི་དང་དབྱངས་གསལ་རྟེན་སྙིང་བཅས་བཟླས་པས་ཕྲེང་བ་བྱིན་གྱིས་བརླབས་པ་དང་།
སྐབས་སུ་ཁ་ཟས་ལའང་བཏབ་གྲུབ་ན་ནུས་པའི་མཐུ་སྐྱེད་པར་གསུངས་སོ།\\
\\
\ru Для благословления четок перед началом рецитации Идама
рецитируй эту мантру вместе с гласными и согласными
и мантрой взаимозависимого происхождения.
Также рецитируй это время от времени над едой, это усилит ее силу и мощь.
\normalsize