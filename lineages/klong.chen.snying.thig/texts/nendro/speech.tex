\section*{Благословение речи}
\\
\ti
ཨོཾ་ཨཱཿཧཱུྂ། ལྕེ་དབང་རྂ་ཡིག་ལས་བྱུང་མེས་བསྲེགས་ནས། \\
།འོད་དམར་རྣམ་པའི་རྡོ་རྗེ་རྩེ་གསུམ་སྦུབས།\\
།ཨཱ་ལི་ཀཱ་ལིའི་མཐའ་སྐོར་རྟེན་འབྲེལ་སྙིང་།\\
།མུ་ཏིག་ཕྲེང་བ་ལྟ་བུའི་ཡིག་འབྲུ་ལས།\\
།འོད་འཕྲོས་རྒྱལ་བ་སྲས་བཅས་མཆོད་པས་མཉེས།\\
།སླར་འདུས་ངག་སྒྲིབ་དག་ནས་གསུང་རྡོ་རྗེའི།\\
།བྱིན་རླབས་དངོས་གྲུབ་ཐམས་ཅད་ཐོབ་པར་འགྱུར།\\
\\
\ru
OM A ХУМ Огнем, возникающим из слога Рам сжигается язык,\\
Возникает трехконечная ваджра из красного света,\\
Санскритские гласные и согласные, окружены мантрой\\
Сущности Взаимозависимого Происхождения,\\
Из слогов подобных жемчужному ожерелью,\\
Распространяется свет, свершающий радующие \\
подношения Буддам и Бодхисаттвам,\\
Собирается обратно, очищает омрачения речи, обретается\\
Все благословение и сиддхи ваджрной речи.

\subsection*{Мантра Гласных}
\\
\ti
ཨ་ཨཱ། ཨི་ཨཱི། ཨུ་ཨཱུ། རྀ་རཱྀ། ལྀ་ལཱྀ། ཨེ་ཨཻ། ཨོ་ཨཽ། ཨཾ་ཨཿ
\ru
7 раз

\subsection*{Мантра Согласных}
\\
\ti
ཀ་ཁ་ག་གྷ་ང་། \\
ཙ་ཚ་ཛ་ཛྷ་ཉ། \\
ཊ་ཋ་ཌ་ཌྷ་ཎ། \\
ཏ་ཐ་ད་དྷ་ན། \\
པ་ཕ་བ་བྷ་མ། \\
ཡ་ར་ལ་ཝ། \\
ཤ་ཥ་ས་ཧ་ཀྵཿ 
\ru
7 раз

\subsection*{Мантра Взаимозависимого Возникновения}
\\
\ti
ཨོཾ་ཡེ་དྷརྨཱ་ཧེ་ཏུ་པྲ་བྷཱ་ཝཱ་ཧེ་ཏུནྟེ་ཥཱནྟ་ཐཱ་ག་ཏོ་ཧྱ་ཝ་དཏ།\\
 ཏེ་ཥཱཉྩ་ཡོ་ནི་རོ་དྷ་ཨེ་ཝྃ་ཝཱ་དཱི་མ་ཧཱ་ཤྲ་མ་ཎཿསྭཱ་ཧཱ།
\ru
7 раз

\subsection*{Мантра Усиления Заслуг}
\\
\ti
ཨོཾ་སམྦྷ་ར་སམྦྷ་ར་བི་མ་ནཱ་སཱ་ར་མ་ཧཱ་ཛམྦྷ་ཧཱུྂ།\\
ཨོཾ་སྨ་ར་སྨ་ར་བི་མཱ་ན་སྐ་ར་མ་ཧཱ་ཛ་བ་ཧཱུྂ།\\
\\
\ru
7 раз\\
\\
\ti
ཨོཾ་རུ་ཙི་ར་མ་ཎི་པྲ་བརྡྷ་ནཱ་ཡེ་སྭཱ་ཧཱ།\\
ཧྲཱིཿབཛྲ་ཛི་ཧཱ་མནྟྲ་དྷ་ར་བརྡྷ་ནི་ཨོཾ།\\
\\
\ru
7 раз
