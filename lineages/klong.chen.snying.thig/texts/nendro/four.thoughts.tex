\newpage
\section{Четыре мысли отвлекающие от самсары}
\\
\subsection{Свободы и преимущества человеческой жизни}
\\
\subsection*{Восемь свобод}
\\
\ti
ད་རེས་དམྱལ་བ་ཡི་དྭགས་དུད་འགྲོ་དང༌།\\
ཚེ་རིང་ལྷ་དང་ཀླ་ཀློ་ལོག་ལྟ་ཅན།\\
སངས་རྒྱས་མ་བྱོན་ཞིང་དང་ལྐུགས་པ་སྟེ།\\
མི་ཁོམས་བརྒྱད་ལས་ཐར་བའི་དལ་བ་ཐོབ།\\
\\
\ru
Родиться в аду, голодным духом или животным\\
Среди богов-долгожителей, дикарем или с неверными взглядами\\
В мире, где Будда не пришел или родиться слабоумным\\
Я свободен от этих восьми неблагоприятных состояний,\\
где нет возможности практиковать Дхарму.\\

\newpage
\subsection*{Пять собственных и пять дарований окружения}
\\
\ti
མིར་གྱུར་དབང་པོ་ཚང་དང་ཡུལ་དབུས་སྐྱེས།\\
ལས་མཐའ་མ་ལོག་བསྟན་ལ་དད་པ་སྟེ།\\
རང་ཉིད་འབྱོར་པ་ལྔ་ཚང་སངས་རྒྱས་བྱོན། \\
ཆོས་གསུངས་བསྟན་པ་གནས་དང་དེ་ལ་ཞུགས།\\
བཤེས་གཉེན་དམ་པས་ཟིན་དང་གཞན་འབྱོར་ལྔ།\\
ཐམས་ཅད་རང་ལ་ཚང་བའི་གནས་ཐོབ་ཀྱང༌།\\
རྐྱེན་མང་ངེས་པ་མེད་པས་ཚེ་སྤངས་ནས།\\
འཇིག་རྟེན་ཕ་རོལ་ཉིད་དུ་སོན་པར་འགྱུར།\\
བློ་སྣ་ཆོས་ལ་སྒྱུར་ཅིག་གུ་རུ་མཁྱེན།\\
ལམ་གོལ་དམན་པར་མ་གཏོང་ཀུན་མཁྱེན་རྗེ།\\
གཉིས་སུ་མེད་དོ་དྲིན་ཅན་བླ་མ་མཁྱེན།\\
\\
\ru
Я родился человеком, с совершенным органами чувств,
в центральной земле. Моя жизнь не вредна или неправильна,
 наполнена верой в учение Будды — Это пять собственных моих дарований.\\
 \\
Будда пришел в этот мир, Он учил Дхарме, его учение сохрани\-лось, я принял учение,
И истинный духовный друг принял меня — 
это пять дарований среды. Хотя я сейчас все это имею,
Так легко упустить это из рук и
Отправится в иные миры.\\
\\
Гуру Ринпоче, обрати мое мышление к Дхарме!
Всеведующие мастера, Лонгченпа и Джигме Лингпа,
не дайте мне свернуть с пути.
Милосердный учитель, единый с ними, помни меня!\\

\newpage
\subsection*{Трудность обретения}
\\
\ti
ད་རེས་དལ་རྟེན་དོན་ཡོད་མ་བྱས་ན།\\
ཕྱི་ནས་ཐར་པ་བསྒྲུབ་པའི་རྟེན་མི་རྙེད།\\
བདེ་འགྲོའི་རྟེན་ལ་བསོད་ནམས་ཟད་གྱུར་ནས།\\
ཤི་བའི་འོག་ཏུ་ངན་སོང་ངན་འགྲོར་འཁྱམས།\\
དགེ་སྡིག་མི་ཤེས་ཆོས་ཀྱི་སྒྲ་མི་ཐོས།\\
དགེ་བའི་བཤེས་དང་མི་མཇལ་མཚང་རེ་ཆེ།\\
སེམས་ཅན་ཙམ་གྱི་གྲངས་དང་རིམ་པ་ལ།\\
བསམས་ན་མི་ལུས་ཐོབ་པ་སྲིད་མཐའ་ཙམ།\\
མི་ཡང་ཆོས་མེད་སྡིག་ལ་སྤྱོད་མཐོང་ན།\\
ཆོས་བཞིན་སྤྱོད་པ་ཉིན་མོའི་སྐར་མ་ཙམ།\\
བློ་སྣ་ཆོས་ལ་སྒྱུར་ཅིག་གུ་རུ་མཁྱེན།\\
ལམ་གོལ་དམན་པར་མ་གཏོང་ཀུན་མཁྱེན་རྗེ།\\
གཉིས་སུ་མེད་དོ་དྲིན་ཅན་བླ་མ་མཁྱེན།\\
\\
\ru
Если я не воспользуюсь сейчас свободами и преимуществами —
я не смогу найти опору для достижения освобождения в будущем.
Когда исчерпаются заслуги, причина этого счастли\-вого существования,
после смерти я стану скитальцем в нижних мирах.
Не различая хорошее от плохого, я не услышу слова Дхармы.
Не встречу духовного друга — как ужасно!\\
\\
Если задуматься о разнообразии живых существ,
Понимаешь насколько мала вероятность обретения человеческого тела.
И даже среди человеческих существ,
— видя как их поведение вредоносно и противоречит Дхарме,
понимаешь, что тех, кто действительно действует согласно Дхарме,
так мало как звед днем на небе.\\
\\
О, Гуру Ринпоче, обрати мое мышление к Дхарме!
Всеведующие мастера, Лонгченпа и Джигме Лингпа,
не дайте мне свернуть с пути.
Милосердный учитель, единый с ними, помни меня!

\newpage
\subsection*{Восемь непредвиденных обстоятельств}
\\
\ti
གལ་ཏེ་མི་ལུས་རིན་ཆེན་གླིང་ཕྱིན་ཡང༌།\\
ལུས་རྟེན་བཟང་ལ་བྱུར་པོ་ཆེ་ཡི་སེམས། \\
ཐར་པ་བསྒྲུབ་པའི་རྟེན་དུ་མི་རུང་ཞིང༌། \\
ཁྱད་པར་བདུད་ཀྱིས་ཟིན་དང་དུག་ལྔ་འཁྲུགས། \\
ལས་ངན་ཐོག་ཏུ་བབས་དང་ལེ་ལོས་གཡེངས། \\
གཞན་ཁོལ་བྲན་གཡོག་འཇིགས་སྐྱོབ་ཆོས་ལྟར་བཅོས། \\
རྨོངས་སོགས་འཕྲལ་བྱུང་རྐྱེན་གྱི་མི་ཁོམ་བརྒྱད། \\
བདག་ལ་ཆོས་ཀྱི་འགལ་ཟླར་ལྷགས་པའི་ཚེ། \\
བློ་སྣ་ཆོས་ལ་བསྒྱུར་ཅིག་གུ་རུ་མཁྱེན། \\
ལམ་གོལ་དམན་པར་མ་གཏོང་ཀུན་མཁྱེན་རྗེ། \\
གཉིས་སུ་མེད་དོ་དྲིན་ཅན་བླ་མ་མཁྱེན། \\
\\
\ru
\noindent Хотя я достиг этого драгоценного острова, человеческого рождения,
изменчивый и страстный ум даже при такой опоре не сможет стать основой достижения освобождения.\\
\\
Особенно, во власти препятсвующих условий или действий пяти ядов;
под натиском неблагой кармы или когда я отвлекся ленью;
как раб под чьим-то контролем или вовлекаюсь в Дхарму из страха;
притворяюсь, что практикую или хронически бесчув\-ственен и глуп
— это восемь непредвиденных обстоятельств которые
делают практику Дхармы невозможной.\\
\\
Когда они приходят ко мне, моя практики Дхармы в опасности!
О, Гуру Ринпоче, обрати мое мышление к Дхарме!
Всеведующие мастера, Лонгченпа и Джигме Лингпа,
не дайте мне свернуть с пути.
Милосердный учитель, единый с ними, помни меня!

\newpage
\subsection*{Восемь несовместимых с Дхармой состояний ума}
\\
\ti
སྐྱོ་ཤས་ཆུང་ཞིང་དད་པའི་ནོར་དང་བྲལ། \\
འདོད་སྲེད་ཞགས་པས་བཅིངས་དང་ཀུན་སྤྱོད་རྩུབ། \\
མི་དགེ་སྡིག་ལ་མི་འཛེམ་ལས་མཐའ་ལོག \\
རིས་ཆད་བློ་ཡི་མི་ཁོམ་རྣམ་པ་བརྒྱད། \\
བདག་ལ་ཆོས་ཀྱི་འགལ་ཟླར་ལྷགས་པའི་ཚེ། \\
བློ་སྣ་ཆོས་ལ་བསྒྱུར་ཅིག་གུ་རུ་མཁྱེན། \\
ལམ་གོལ་དམན་པར་མ་གཏོང་ཀུན་མཁྱེན་རྗེ། \\
གཉིས་སུ་མེད་དོ་དྲིན་ཅན་བླ་མ་མཁྱེན། \\
\\
\ru
Слабое отречение, отсуствие драгоценности веры,
Пленение арканом привязанности, склонность к развращенному поведе\-нию.
Наслаждение неблагим, вредоносные действия, отутствие толики интереса,
Преступление обетов и нарушение самай, —
Это восемь несовместимых с Дхармой состояний ума.\\
\\
Когда они приходят ко мне, моя практики Дхармы в опасности!\\
\\
О, Гуру Ринпоче, обрати мое мышление к Дхарме!
Всеведующие мастера, Лонгченпа и Джигме Лингпа, не дайте
мне свернуть с пути. Милосердный учитель, единый с ними, помни меня!

\newpage
\subsection{Непостоянство жизни}
\ti
ད་ལྟ་ནད་དང་སྡུག་བསྔལ་གྱིས་མ་གཟིར།\\
བྲན་ཁོལ་ལ་སོགས་གཞན་དབང་མ་གྱུར་པས།\\
རང་དབང་ཐོབ་པའི་རྟེན་འབྲེལ་འགྲིག་དུས་འདིར།\\
སྙོམས་ལས་ངང་དུ་དལ་འབྱོར་ཆུད་གསོན་ན།\\
འཁོར་དང་ལངོ ས་སྤྱོད་ཉེ་དུ་འབྲེལ་བ་ལྟ།\\
ལྟ་ཅི་གཅེས་པར་བཟུང་བའི་ལུས་འདི་ཡང་།\\
མལ་གྱི་ནང་ནས་ས་ཕྱོགས་སྟོང་པར་བསྐྱལ།\\
ཝ་དང་བྱ་རྒོད་ཁྱི་ཡིས་འདྲད་པའི་དུས།\\
བར་དོའི་ཡུལ་ན་འཇིགས་པ་ཤིན་ཏུ་ཆེ།\\
བློ་སྣ་ཆོས་ལ་བསྒྱུར་ཅིག་གུ་རུ་མཁྱནེ།\\
ལམ་གོལ་དམན་པར་མ་གཏོང་ཀུན་མཁྱནེ་རྗེ།\\
གཉིས་སུ་མེད་དོ་དྲིན་ཅན་བླ་མ་མཁྱེན།\\
\\
\ru
Сейчас я не измучен страданием и болезнями,
И не попал под власть других подобно рабу.\\
\\
Поэтому, обладая столь совершенно-успешным качеством не\-зависимости,
Если я в бесцельности растрачу человеческую жи\-знь через свою собственную лень,
В час, когда придется оста\-вить близких, богатсво, родных и любимых,
Когда это тело которое я так лелеял,
Перекочует из своей постели в пустынное место
На растерзание лисам, стервятникам и псам,
Тогда в бардо не будет ничего кроме террора и страха.\\
\\
О, Гуру Ринпоче, обрати мое мышление к Дхарме!
Всеведующие мастера, Лонгченпа и Джигме Лингпа,
не дайте мне свернуть с пути. Милосердный учитель,
единый с ними, помни меня!

\newpage
\subsection{Закон причины и следствия безошибочен}
\\
\ti
དགེ་སྡིག་ལས་ཀྱི་རྣམ་སྨིན་ཕྱི་བཞིན་འབྲང་།\\
\\
\ru
Следствия благих и неблагих действий всегда следуют за мной.

\subsection{Изьяны самсары}

\subsection*{Страдание в горячих адах}
\\
\ti
ཁྱད་པར་དམྱལ་བའི་འཇིག་རྟེན་ཉིད་སོན་ན།\\
ལྕགས་བསྲེགས་ས་གཞིར་མཚོན་གྱིས་མགོ་ལུས་འདྲལ།\\
སོག་ལེས་གཤོགས་དང་ཐོ་ལུམ་འབར་བས་འཚིར།\\
སྒོ་མེད་ལྕགས་ཁྱམི་འཐུམས་པར་འོ་དོད་འབོད།\\
འབར་བའི་གསལ་ཤིང་གིས་འབུགས་ཁྲོ་ཆུར་འཚོད།\\
ཀུན་ནས་ཚ་བའི་མེས་བསྲེགས་བརྒྱད་ཚན་གཅིག\\
\\
\ru
Если я рожусь существом адов
На поверности из лавы буду изранен и обезглавлен
Рассечен пилами и раздроблен раскаленными молотами
В ужасе буду вопить, заключенный в железных домах без дверей Пронзенный каленными копьями и свареный в лаве,
Выжженый яростным пламенем я познаю ужас восьми горячих адов.

\newpage
\subsection*{Cтрадание в холодных адах}
\\
\ti
གངས་རི་སྟུག་པོའི་འདབས་དང་ཆུ་འཁྱགས་ཀྱི། \\
གཅོང་རོང་ཡ་ངའི་གནས་སུ་བུ་ཡུག་སྦྲེབས། \\
གྲང་རེག་རླུང་གིས་བཏབ་པའི་ལང་ཚོ་ནི། \\
ཆུ་བུར་ཅན་དང་ལྷག་པར་བརྡོལ་བ་ཅན། \\
སྨྲེ་སྔགས་རྒྱུན་མི་ཆད་པར་འདོན་པ་ཡང་། \\
ཚོར་བའི་སྡུག་བསྔལ་བརྣག་པར་དཀའ་བ་ཡིས།\\
ཟུངས་ཀྱིས་རབ་བཏང་འཆི་ཁའི་ནད་པ་བཞིན།\\
ཤུགས་རིང་འདོན་ཅིང་སོ་ཐམ་པགས་པ་འགས། \\
ཤའུ་ཐོན་ནས་ལྷག་པར་འགས་ཏེ་བརྒྱད།\\
\\
\ru
Где вершины гор, на обрывах льда
Посреди страшных мест, где метель и снег
Мое нежное тело встречает замораживающие ветра
Превращаясь в волдыри и гнойные язвы
Непрерывный вопль и мучительный крик
Немыслимые страдания
Человека, которого покидают силы
Я дышу глубоко и прерывисто сжав зубы, а кожа трескается.
И моя плоть разлагается заживо в глубоких ранах,
не давая мне умереть Так я познаю отчаяние восьми холодных адов.

\newpage
\subsection*{Cтрадание в промежуточных адах}
\\
\ti
དེ་བཞིན་སྤུ་གྲིའི་ཐང་ལ་རྐང་པ་གཤོགས།\\
རལ་གྲིའི་ཚལ་དུ་ལུས་ལ་བཅད་གཏུབས་བྱེད།\\
རོ་མྱགས་འདམ་ཚུད་ཐལ་ཚན་རབ་མེད་ཀློང་། \\
མནར་བའི་ཉེ་འཁོར་བ་དང་འགྱུར་བ་ཅན།\\
\\
\ru
Мои ступни разрезаются на ленты в Равнине Бритвенных лезвий,
в Лесу Мечей мое тело располасывается ножами\\
Я погружаюсь в Болото Гниющих Трупов и в Яму Горящих Углей\\
Так я познаю мучения Смежных Адов.

\subsection*{Cтрадание в неопределенных адах}
\\
\ti
སྒོ་དང་ཀ་བ་ཐབ་དང་ཐག་པ་སོགས།\\
རྟག་ཏུ་བཀོལ་ཞིང་སྤྱོད་པའི་ཉི་ཚེ་བ། \\\
རྣམ་གྲངས་བཅོ་བརྒྱད་གང་ལས་འབྱུང་བའི་རྒྱུ། \\
ཞེ་སྡང་དྲག་པོའི་ཀུན་སློང་སྐྱེས་པའི་ཚེ། \\
བློ་སྣ་ཆོས་ལ་བསྒྱུར་ཅིག་གུ་རུ་མཁྱནེ ། \\
ལམ་གོལ་དམན་པར་མ་གཏོང་ཀུན་མཁྱནེ ་རྗེ། \\
གཉིས་སུ་མེད་དོ་དྲིན་ཅན་བླ་མ་མཁྱེན།\\
\\
\ru
Используемый как двери, колонны, печи, веревки и прочее\\
Я познаю рабство изменчивого ада\\
Причина рождения в восем\-надцати адах —\\
Сильный гнев и ненависть.\\
О, Гуру Ринпоче, обрати мое мышление к Дхарме!\\
Всеведующие мастера, Лонг\-ченпа и Джигме Лингпа,\\
не дайте мне свернуть с пути.\\
Милосердный учитель, единый с ними, помни меня!
\\

\newpage
\subsection*{Cтрадания в мире голодных духов}
\\
\ti
དེ་བཞིན་ཕོངས་ལ་ཉམས་མི་དགའ་བའི་ཡུལ།\\
བཟའ་བཏུང་ལངོ ས་སྤྱོད་མིང་ཡང་མི་གྲགས་པར། \\
ཟས་སྐོམ་ལོ་ཟླར་མི་རྙེད་ཡི་དྭགས་ལུས། \\
རིད་ཅིང་ལྡང་བའི་སྟོབས་ཉམས་རྣམ་པ་གསུམ། \\
གང་ལས་འབྱུང་བའི་རྒྱུ་ནི་སེར་སྣ་ཡིན།\\
\\
\ru
Точно также в мрачной и обездоленной области
Где про пищу, воду и комфорт не слыхать и подавно
Преты не могут найти ничего поесть и попить в течении месяцев и лет.
Ихнии тела настолько истощены что нету сил даже стоять. Они страдают
От трех видов загрязнений и являются результатом перерожде\-ния от жадности.

\subsection*{Cтрадания в мире животных}
\\
\ti
གཅིག་ལ་གཅིག་བཟའ་གསོད་པའི་འཇིགས་པ་ཆེ།\\
བཀོལ་ཞིང་སྤྱོད་པས་ཉམ་ཐག་བླང་དོར་རྨོངས། \\
ཕ་མཐའ་མེད་པའི་སྡུག་བསྔལ་གྱིས་གཟིར་བའི། \\
ས་བོན་གཏི་མུག་མུན་པར་འཁྱམས་པ་བདག \\
བློ་སྣ་ཆོས་ལ་བསྒྱུར་ཅིག་གུ་རུ་མཁྱནེ ། \\
ལམ་གོལ་དམན་པར་མ་གཏོང་ཀུན་མཁྱནེ ་རྗེ། \\
གཉིས་སུ་མེད་དོ་དྲིན་ཅན་བླ་མ་མཁྱེན།\\
\\
\ru
В постоянном страхе быть съеденными другими,
Изнуренные рабством, не ведая, что принимать, а что отвергать,
Животные поглощены бесконечным страданием,
Суть которого упрямая тупость, — когда я попадаю в этот мрак неведения,
О, Гуру Ринпоче, обрати мое мышление к Дхарме!
Всеведующие масте\-ра, Лонгченпа и Джигме Лингпа,
не дайте мне свернуть с пути.
Милосердный учитель, единый с ними, помни меня!

\section*{Осознание собственных ошибок}
\subsection*{Три Колесницы}
\\
\ti
ཆོས་ལམ་ཞུགས་ཀྱང་ཉེས་སྤྱོད་མི་སྡོམ་ཞིང་།\\
ཐེག་ཆེན་སྒོར་ཞུགས་གཞན་ཕན་སེམས་དང་བྲལ།\\
དབང་བཞི་ཐོབ་ཀྱང་བསྐྱེད་རྫོགས་མི་སྒོམ་པའི།\\
ལམ་གོལ་འདི་ལས་བླ་མས་བསྒྲལ་དུ་གསོལ།\\
\\
\ru
Хотя я вступил на путь Дхармы, я не отрекся от ошибочных путей.
Хотя я вошел во врата Махаяны, я лишен намерения принести благо другим.
Хотя я получил четыре посвящения, я не практикую стадии зарождения и завершения.
О, Учитель, освободи меня от этого ошибочного пути!

\subsection*{Воззрение, Медитация и Поведение}
\\
\ti
ལྟ་བ་མ་རྟོགས་ཐོ་ཅོའི་སྤྱོད་པ་ཅན།\\
སྒོམ་པ་ཡངེ ས་ཀྱང་གོ་ཡུལ་འུད་གོག་\\
འཐག སྤྱོད་པ་ནོར་ཀྱང་རང་སྐྱོན་མི་སེམས་པའི། \\
ཆོས་དྲེད་འདི་ལས་བླ་མས་བསྒྲལ་དུ་གསོལ།\\
\\
\ru
Хотя я не реализовал Воззрение, я веду себя как Йогин Безумной Мудрости,
Без устойчивости в Медитации я позволяю себе застрять в концепциях и игре ума,
Хотя мое Поведение ошибочно, я виню не себя.
О, Учитель, освободи меня от высокомерия, упрямства и безчувсственности!

\newpage
\subsection*{Отвлечение от настоящей жизни}
\\
\ti
ནང་པར་འཆི་ཡང་གནས་གོས་ནོར་ལ་སྲེད།\\
ན་ཚོད་ཡོལ་ཡང་ངེས་འབྱུང་སྐྱོ་ཤས་བྲལ། \\
ཐོས་པ་ཆུང་ཡང་ཡོན་ཏན་ཅན་དུ་རློམ། \\
མ་རིག་འདི་ལས་བླ་མས་བསྒྲལ་དུ་གསོལ།\\
\\
\ru
Хотя я могу умереть завтра, я цепляюсь за жилье одежду и богатство.
Хотя я уже не молод, я лишен разочерование и отречения от Самсары.
Познав крупицу Дхармы, кичусь ученостью направо и налево.
О, Учитель, освободи меня от подобной тупости!

\subsection*{Восемь мирских забот}
\\
\ti
རྐྱེན་ཁར་འཆོར་ཡང་འདུ་འཛི་གནས་སྐོར་སེམས།\\
དབེན་པ་བརྟེན་ཀྱང་རང་རྒྱུད་ཤིང་ལྟར་རེངས། \\
དུལ་བར་སྨྲ་ཡང་ཆགས་སྡང་མ་ཞིག་པའི། \\
ཆོས་བརྒྱད་འདི་ལས་བླ་མས་བསྒྲལ་དུ་གསོལ། \\
གཉིད་འཐུག་འདི་ལས་མྱུར་དུ་སད་དུ་གསོལ། \\
ཁྲི་མུན་འདི་ལས་མྱུར་དུ་དབྱུང་དུ་གསོལ། \\
ཞེས་འབོད་པ་དྲག་པོས་ཐུགས་རྗེ་བསླང་བར་བྱའོ།\\
\\
\ru
Хотя я могу подвернуться опасностям, я ищу Дхарма-общение
в публичных местах, думая что это паломничество.
Хотя пребы\-ваю в ретрите, мышление словно посох, лишено гибкости.
По\-хваляясь покоем и тихой речью, я не избавился от гнева и жадности.
О, Учитель, защити меня от восьми мирских целей.
Пробуди меня от этого глубокого сна неведения.
Вызволи меня из этого мрачного самозаключения!
