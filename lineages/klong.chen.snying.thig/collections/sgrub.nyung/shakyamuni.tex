\section{Апам Тертон.\\Очень краткий ритуал Будды Шакьямуни\\из откровений Сердца.}
\scriptsize
Из цикла тайной Дхармы Красной Тары\\
Необходимая для отбрасывания негативностей \\
Глубокая йога Могучего Победителя, Льва Шакьев\\
Кратко исполняется так.\\
Мандалу, опоры Тела, Речи и Ума,\\
Внешние и внутренние подношения размести соответственно.\\
Приход к Прибежищу, зарождение просветленного ума и семь ветвей.\\
\\
\vspace{1cm}
\normalsize
[ \scriptsize Из “Объединённой практики Трёх корней Гуру”: \normalsize
\subsection*{Приход к Прибежищу}
\\

НАМО\\
К Трём Драгоценностям, к божествам подлинных Трёх Корней,\\
Я и другие, все скитальцы, вплоть до обретения Просветления,\\
В состоянии недвойственности \\ \indent самовозникших Тела, Речи и Ума\\
Изначального отсутствия единения и разделения \\ \indent приходим к Прибежищу.\\
\scriptsize
\indent Повтори 3 раза.
\normalsize

\subsection*{Зарождение ума Просветления}
\\
Бессчётные, как небо скитальцы в океане бытия\\
Беспомощны и связаны обманчивыми проявлениями и кармой.\\
Устанавливаю Намерение Трёх Тел в чистой природе, \\
Зарождаю Просветлённый ум \\ \indent как великое самоосвобождение без усилий. \\
\scriptsize
\indent Повтори 3 раза.\\
\normalsize

\subsection*{Семь ветвей}
\\

ХО\\
Перед океаном Трёх Драгоценностей и Корней простираюсь.\\
Чувственные удовольствия, без отбрасывания, \\ \indent как великое блаженство подношу.\\
Проступки и падения в состоянии лишённом природы отвергаю.\\
Сорадуюсь Дхарме самоосвобождения трёх колесниц.\\
Поверните колесо Дхармы проявленной природы для тех, \\ \indent кого нужно укротить.\\
Пребывайте неизменно в состоянии Ваджра светоносной ясности.\\
Незапятнанные корни добродетели посвящаю вне трёх сфер.
\scriptsize
Ритуал зарождения подобен внешней практике
\normalsize ]
\\
\\
Ах\\
Все дхармы изначально лишены точки опоры.\\
Состояние с тремя (вратами) освобождениями.\\
На лотосе, поверх трона из солнца и луны,\\
Сам ум – Тело Льва Шакьев\\
Сияет как очищенное золото;\\
Лик мирный с улыбкой;\\
Завиток урна излучает пятицветный свет;\\
Две руки: левая попирает землю,\\
Правая – в жесте равновесия, ноги сложены в (ваджрной) позе;\\
Носит три дхармовые одеяния, сверкает великолепием;\\
Полностью украшен знаками и признаками.\\
Из трёх мест сердечная сила лучей \\
Снопами света во все трёхтысячемерные (миры проникает),\\
В необъятных землях десяти направлений\\
Благородные будды и бодхисаттвы\\
Появляются в Телах Царя Шакьев.\\
\newpage
\subsection*{Призывание}
\\
Дэва, ставший Защитником всех разумных существ без исключений,\\
Разбивший непобедимые орды демонов и их армии,\\
Познавший все без исключения вещи, как они есть,\\
Бхагаван со свитой, молю, приди в это место.\\
\\
\ti
%badz+ra sa ma yA dza:dzaHhU~MbaM ho: pad+ma ka ma la ye stwaM:
%oM ar+g+haM pad+yaM puSh+pe d+hup+pe a lo ke gan+d+he nai wit+ye shab+ta pra tits+tsha ye swA hA:
བཛྲ་ས་མ་ཡཱ་ཛ༔ ཛཿཧཱུྃབཾ་ཧོ པདྨ་ཀ་མ་ལ་ཡེ་སྟྭཾ༔ \\
ཨོཾ་ཨརྒྷཾ་པདྱཾ་པུཥྤེ་དྷུཔྤེ་ཨ་ལོ་ཀེ་གནྡྷེ་ནཻ་ཝིཏྱེ་ཤབྟ་པྲ་ཏིཙྪ་ཡེ་སྭཱ་ཧཱ༔ \\
\\
\ru
БАДЗРА САМАЯ ДЗА ДЗАХ ХУМ БАМ ХОХ \\
ПАДМА КАМАЛАЕ СТВАМ\\
ОМ АРГХАМ ПАДЬЯМ ПУШПЕ ДЮППЕ\\
АЛОКЕ ГАНДХЕ НЕЙВИТЬЕ ШАБДА ПРАТИЦЦХАЕ СВАХА\\
\\
Могущественный Святой, Защитник трёх миров,\\
Подобный сиянию тысячи солнц золотой горы Сумеру.\\
Тело, излучающее свет знаков и признаков, переполняет счастьем глядящего.\\
Несравненный в трёх существованиях, лев среди людей, склоняюсь пред Тобой!

\subsection*{Затем рецитация}
В сердце, на лунном диске\\
Золотой слог ОМ излучает пятицветный свет.\\
Вокруг него гирлянда мантры вращается вправо.\\
Испусканием  и собиранием света осуществляются два блага, и\\
Обретаются высшие и обычные достижения.\\
\\
Мантра приближения:\\
\\
% tad+ya thA:_oM mu ne mu ne ma hA mu ne shAkya mu ne ye swA hA:\\
\ti ཏདྱ་ཐཱ༔ ཨོཾ་མུ་ནེ་མུ་ནེ་མ་ཧཱ་མུ་ནེ་ཤཱཀྱ་མུ་ནེ་ཡེ་སྭཱ་ཧཱ༔\\
% в одну строку
\noindent \ru \\
ТАДЬЯТА ОМ МУНЕ МУНЕ МАХАМУНЕ ШАКЬЯМУНЕЙЕ СВАХА\\
\scriptsize Повторяй, сколько можешь, затем \normalsize

\newpage
Гирлянда мантры превращается \\ \indent в дхарани взаимозависимого происхождения,\\
Испусканием и собиранием пламени золотого цвета\\
Осуществляются два блага и устраняются все препятствия.\\
Явленное и существующее превращается \\ \indent в мандалу божеств, мантр и дхарматы.\\
\\
\ti
%oM ye d+har+mA he tu pra b+ha wA he tun te Shan+ta thA ga to ha+ya wa data
%te Shany+tsa yo ni ro d+hae waM bA dI ma hA shrA ma Na swA hA
ཨོཾ་ཡེ་དྷརྨཱ་ཧེ་ཏུ་པྲ་བྷ་ཝཱ་ཧེ་ཏུན་ཏེ་ཥནྟ་ཐཱ་ག་ཏོ་ཧྱ་ཝ་དཏ།\\
ཏེ་ཥཉྩ་ཡོ་ནི་རོ་དྷཨེ་ཝཾ་བཱ་དཱི་མ་ཧཱ་ཤྲཱ་མ་ཎ་སྭཱ་ཧཱ །\\
\\
\ru ОМ Е ДХАРМА ХЕТУ ПРАБХАВА \\
ХЕТУНТЕЩАН ТАТХАГАТО \\
ХАЙАВАДАТ ТЕЩАНЬЦАЙО НИРОДХАЕ\\
ЭВАМВАДИ МАХАШРАМАНА СВАХА\\
\\
\scriptsize
Так повторяй сколько нужно. В завершение – восхваление, посвящение и молитвы.\\
Из океана глубоких откровений Лотосорожденного\\
Когда пришло время (исполниться) молитвам счастливчиков,\\
Всеисполняющая драгоценность блага Учения и скитальцев\\
Невыразимая добродетель благодати,\\
Этот особая (практика) Проводника благой кальпы для\\
Для меня и других, которые должны быть усмирены великой добротой\\
Есть несравненные тысячи лучей блага и счастья,\\
Всегда осуществляй памятованием и верой!\\
Слово это охраняет Дрегдже Барва (Пылающий Демон)\\
САМАЯ Печать Печать Печать \\
Печать откровения \\
Печать сокрытия \\
Печать заклинания \\
ГУХЬЯ Знак растворения\\
\\
По просьбе Хранителя Дхармы Шичен Чоктрул Кунзанг Донток Сангак Тенпи Ньима 
в уединённой обители Геньен Драклха Гонпо (это) получил определённо 
Оргьен Тринлей Лингпа, и сам проситель записал. САРВА МАНГАЛАМ\\
\normalsize
\newpage
\subsection*{Посвящение и молитва}
\vspace{1cm}
\subsubsection*{[ Из “Объединённой практики Трёх корней Гуру”}\\
\\
Совершенно чистые корни добродетели от самадxи зарождения и завершения\\
Трёх корней и (накопленные) в трёх временах запечатываю и посвящаю,\\
Чтобы все скитальцы бессчётные, как небо, стали буддами.\\
Пусть все быстро обретут высшее Просветление. ]\\
\\
\subsubsection*{[ Острие удачи}\\
\\
Благоприятствие всей удачи – \\
Божества удачи десяти направлений,\\
Пред вами простираюсь для удачи.\\
Пусть всегда пылает слава удачи!\\
\scriptsize
По просьбе ученика Церинга записал Чойинг Дордже.
\normalsize
 ]\\
\\
