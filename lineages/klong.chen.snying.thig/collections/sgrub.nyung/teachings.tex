\section{Перед началом учений или посвящения}

\subsection{Семь ветвей накопления заслуг}

Я простираюсь перед Ваджрным Учителем\\
И славными Буддами, пребывающими в трех временах.\\
Я прихожу за Прибежищем с недвойственным умом\\
К ставшим полем обучения, Трем Драгоценностям.\\
Молю принять эти чистые подношения,\\
Как реальные, так и воображаемые.\\
Раскаиваюсь во всех, без исключения, пороках и\\
Ошибках, пресекающих поток достижений.\\
Сорадуюсь, практикуя непривязанность,\\
Дхарме, чистоте трех сфер и десяти направлений.\\
Зарождаю ум полного Просветления,\\
Четыре предела чистоты без загрязнений.\\
Подношу свое тело с тройной чистотой\\
Могущественным сугатам и бодхисаттвам.\\
Всю добродетель, собранную в веренице жизней,\\
Я посвящаю Великому Просветлению.

\subsection{Подношение семичастного мандала}

Я подношу эту землю, \\
\indent окропленную благовонной водой и усыпанную цветами,\\
Гору Сумеру, четыре континента, украшенную солнцем и луной,\\
Представив как Поле Будды,\\
Чтобы все скитальцы насладились Чистой Землей.\\
\\
\ti 
ཨི་དཾ་གུ་རུ་རཏྣམཎྜ་ལ་ཀཾ་ནི༄༅་ཏ་ཡཱ་མི།\\
\\
\ru ИДАМ ГУПУ РАТНА МАНДАЛА КАМ НИРЬЯТАЯ МИ

\subsection{Просьба повернуть колесо Учения}

Согласно помыслам разумных существ и\\
В соответствии с их различными способностями,\\
Молю, поверни Колесо Дхармы\\
Великой, Малой и обычной колесниц.