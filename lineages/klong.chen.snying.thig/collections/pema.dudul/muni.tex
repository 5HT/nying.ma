\normalsize
\section{Мипам Ринпоче \\Сокровищница Благословений:\\Садхана Будды Шакьямуни }
\\
\vspace{1cm}

\subsection{Призывание}

\ti ན་མོ་གུ་རུ་ཤཱཀྱ་མུ་ན་ཡེ།\\
\\
\ru НАМО ГУРУ ШАКЬЯМУНАЕ\\
\\
\scriptsize
\ti
དེ་ཡང་མདོ་ཏིང་འཛིན་རྒྱལ་པོ་ལས། 
འཆག་དང་འདུག་དང་འགྲེང་དང་ཉལ་བ་ན། 
།མི་གང་ཐུབ་པའི་བཟླ་དྲན་བྱེད་པ། 
།དེ་ཡི་མདུན་ན་རྟག་ཏུ་སྟོན་པ་བཞུགས། 
།དེ་ནི་རྒྱ་ཆེན་མྱ་ངན་འདའ་བར་འགྱུར། 
ཞེས་དང་། 
སྐུ་ལུས་དག་ནི་གསེར་གྱི་མདོག་འདྲ་བས། 
།འཇིག་རྟེན་མགོན་པོ་ཀུན་ནས་རབ་ཏུ་མཛེས། 
དམིགས་པ་འདི་ལ་གང་གི་སེམས་འཇུག་པ། 
།བྱང་ཆུབ་སེམས་དཔའ་དེ་ནི་མཉམ་བཞག་ཡིན། 
ཞེས་གསུངས་པ་བཞིན་དུ། 
བདག་ཅག་རྣམས་ཀྱི་སྟོན་པ་མཚུངས་པ་མེད་པ་ཐུབ་པའི་དབང་པོ་རྗེས་སུ་དྲན་པའི་རྣལ་འབྱོར་དུ་བྱ་བ་ནི། 
འདི་ལྟ་སྟེ།\\
\\
\ru
В сутре о Царе Самадхи сказано:
“Перед любым человеком, который памятует луноподобного Шакьямуни,
Когда ходит, сидит, стоит или спит,
Учитель пребывает всегда,
Пока тот не обретет Великую нирвану.”
И далее:
“Его чистое Тело цветом подобно золоту.
Защитник мира, полностью прекрасный во всем.
Любой, хранящий этот образ в уме,
Пребывает в созерцании бодхисаттв”.
Согласно сказанному, способ выполнения йоги памятования нашего несравненного Учителя,
Повелителя мудрецов, таков.\\
\\
\normalsize
\newpage
\subsection{Прибежище и бодхичитта}
\vspace{0.5cm}
\ti
སངས་རྒྱས་ཆོས་དང་ཚོགས་ཀྱི་མཆོག་རྣམས་ལ།\\
བྱང་ཆུབ་བར་དུ་བདག་ནི་སྐྱབས་སུ་མཆི།\\
བདག་གི་སྦྱིན་སོགས་བགྱིས་པའི་བསོད་ནམས་ཀྱིས།\\
འགྲོ་ལ་ཕན་ཕྱིར་སངས་རྒྱས་འགྲུབ་པར་ཤོག\\
\\
\ru
К Будде, Дхарме и Высшему собранию\\
Вплоть до достижения Пробуждения\\
я прихожу за Прибежищем.\\
Пусть благодаря заслуге от деяний даяния и других\\
Я обрету Просветление для пользы скитальцев.\\
\\
Три раза.\\

\subsection{Созерцание четырех безмерных}
\vspace{0.5cm}
\ti
སེམས་ཅན་ཐམས་ཅད་བདེ་བ་དང་བདེ་བའི་རྒྱུ་དང་ལྡན་པར་གྱུར་ཅིག\\
སྡུག་བསྔལ་དང་སྡུག་བསྔལ་གྱི་རྒྱུ་དང་བྲལ་བར་གྱུར་ཅིག\\
སྡུག་བསྔལ་མེད་པའི་བདེ་བ་དང་མི་འབྲལ་བར་གྱུར་ཅིག\\
ཉེ་རིང་ཆགས་སྡང་གཉིས་དང་བྲལ་བའི་བཏང་སྙོམ་ཚད་མེད་པ་ལ་གནས་པར་གྱུར་ཅིག\\
\\
\ru
Пусть все разумные существа\\
Обретут счастье и причины счастья.\\
Да будут они свободны от страдания и причины страдания.\\
Да не будут они отделены от счастья без страдания.\\
Пусть пребывают в безмерной равностности, свободной от\\
двойственности ближнего и дальнего, принятия и отвержения.\\
\\
Три раза

\newpage
\subsection{Визуализация}
\scriptsize
\ti
ཆོས་ཐམས་ཅད་སྣང་ལ་རང་བཞིན་མ་གྲུབ་པའི་དོན་ཡིད་ལ་དྲན་པའི་ངང་ནས།\\
\ru
Удерживай в уме смысл того, что все дхармы проявляются,
как не имеющие собственной природы. В этом состоянии произноси:\\
\\
\\
\normalsize
\ti
ཨཿ སྐྱེ་མེད་སྟོང་པ་ཉིད་དང་རྟེན་འབྱུང་གི།\\
སྣང་བ་འགག་མེད་ཟུང་འཇུག་སྒྱུ་མའི་ཚུལ།\\
རང་མདུན་ནམ་མཁར་མཆོད་སྤྲིན་རྒྱ་མཚོའི་དབུས།\\
རིན་ཆེན་སེང་ཁྲི་པད་ཉི་ཟླ་བའི་སྟེང་།\\
སྟོན་པ་མཚུངས་མེད་ཤཱཀྱ་སེང་གེ་ནི།\\
གསེར་གྱི་མདོག་ཅན་མཚན་དང་དཔེ་བྱད་ལྡན།\\
ཆོས་གོས་གསུམ་གསོལ་རྡོ་རྗེ་སྐྱིལ་ཀྲུང་བཞུགས།\\
ཕྱག་གཡས་ས་གནོན་ཕྱག་རྒྱ་ལེགས་བརྐྱངས་ཤིང་།\\
ཕྱག་གཡོན་མཉམ་བཞག་བདུད་རྩིའི་ལྷུང་བཟེད་བསྣམས།\\
གསེར་གྱི་རི་ལྟར་གཟི་བརྗིད་དཔལ་འབར་བ།\\
ཡེ་ཤེས་འོད་ཟེར་དྲ་བས་མཁའ་དབྱིངས་ཁྱབ།\\
ཉེ་བའི་སྲས་བརྒྱད་གནས་བརྟན་བཅུ་དྲུག་སོགས། \\
འཕགས་ཚོགས་རྒྱ་མཚོའི་འཁོར་གྱིས་ཡོངས་བསྐོར་ཞིང་།\\
དྲན་པ་ཙམ་གྱིས་སྲིད་ཞིའི་མཐའ་གཉིས་ལས། \\
རྣམ་གྲོལ་བདེ་བ་མཆོག་གི་དཔལ་སྩོལ་བ། \\
སྐྱབས་ཀུན་འདུས་པའི་བདག་ཉིད་ཆེན་པོར་གསལ།\\
\\
\ru
АХ Нерожденная пустотность и беспрепятственное проявление\\
Взаимозависимого происхождения \\
соединяются волшебным образом —\\
Передо мной в небе посреди океана облаков подношений\\
На драгоценном львином троне, лотосе и луне\\
Пребывает несравненный Учитель, Лев Шакьев:\\
\newpage
Золотого цвета, обладающий знаками и признаками,\\
На нем три одеяния Дхармы, сидит,\\
Скрестив ноги в позе ваджры\\
Его правая рука изящно вытянута в мудре касания земли,\\
Левая рука в мудре равновесия\\
Держит чашу для подаяний с нектаром.\\
Его благородное величие сияет подобно золотой горе.\\
Сеть лучей света Изначального Знания\\
Пронизывает небесную сферу.\\
Окружен свитой океаноподобного благородного собрания —\\
Восемь близких Сыновей, шестнадцать Старейшин и другие.\\
Лишь мысль о нем полностыо освобождает от двух крайностей,
Сансары и нирваны, и дарует славу высшего счастья.\\
Созерцаю Великого Повелителя, собрание всех Прибежищ.\\
\\
\scriptsize
\ti
ཞེས་དེ་ལྟར་སངས་རྒྱས་ཀྱི་སྐུ་ལ་དམིགས་ཏེ་དེ་ན་དངོས་སུ་བཞུགས་ཡོད་སྙམ་པའི་སེམས་བསྐྱེད་མ་ཐག་ཏུ། 
སངས་རྒྱས་རྣམས་ཀྱི་ཡེ་ཤེས་ཀྱི་སྐུ་ལ་ཕྱོགས་དང་དུས་གང་དུ་ཡང་ཉེ་རིང་མི་མངའ་བའི་ཕྱིར། 
གང་དུ་དམིགས་པ་དེ་ཉིད་དུ་ངེས་པར་བཞུགས་པར་འགྱུར་ཏེ། མདོ་ལས། གང་ཞིག་སངས་རྒྱས་ཡིད་བྱེད་པ། 
།དེ་ཡི་མདུན་ན་དེ་བཞུགས་ཏེ། །རྟག་པར་བྱིན་གྱིས་རློབས་བྱེད་ཅིང་། 
།ཉེས་པ་ཀུན་ལས་རྣམ་པར་གྲོལ། ཞེས་གསུངས་ཤིང་། 
རྒྱལ་བ་དེ་ཉིད་ལ་དམིགས་ནས་ཚོགས་བསགས་པ་ཡང་མི་ཟད་པའི་དགེ་རྩ་ཆུད་མི་ཟ་བ་ཡིན་ཏེ། 
ཕལ་པོ་ཆེ་ལས། །རྒྱལ་བ་དེ་དག་ཐོས་མཐོང་མཆོད་པ་བྱས་པས་ཀྱང་།
།ཚད་མེད་པ་ཡི་བསོད་ནམས་ཕུང་པོ་འཕེལ་བར་འགྱུར། 
།ཉོན་མོངས་འཁོར་བའི་སྡུག་བསྔལ་ཐམས་ཅད་སྤང་འགྱུར་དུ། 
།འདུས་བྱས་འདི་ནང་བར་མ་དོར་ནི་ཟད་མི་འགྱུར། ཞེས་དང་། 
དེའི་མདུན་དུ་སྨོན་ལམ་ཇི་ལྟར་བཏབ་པ་ཡང་དེ་བཞིན་དུ་འགྲུབ་སྟེ། 
འཇམ་དཔལ་ཞིང་གི་ཡོན་ཏན་བསྟན་པ་ལས། །ཆོས་རྣམས་ཐམས་ཅད་རྐྱེན་བཞིན་ཏེ། 
།འདུན་པའི་རྩེ་ལ་རབ་ཏུ་གནས། །གང་གིས་སྨོན་ལམ་ཅི་བཏབ་པ། 
།དེ་འདྲའི་འབྲས་བུ་ཐོབ་པར་འགྱུར། ཞེས་གསུངས་པའི་ཚུལ་རྣམས་ལ་ངེས་པ་བརྟན་པོ་བསྐྱེད་དེ།\\
\\
\ru
Так представляй тело Будды. Поскольку для Тела Изначального Знания Будды не существует направления,
времени или расстояния, то как только в уме появится ощущение, что Будда присутствует,
он действительно будет находится, где бы его не представлять. В сутре сказано:
“Кто бы не подумал о Будде,
Перед тем он будет пребывать,
Постоянно даруя благодать
И освобождая от всех ошибок”.
Также сказано:
“Если (созерцаешь) образ самого Победителя, то накапливается заслуга -корень неисчерпаемой
добродетели не растрачивается”.
В сутре о Великом Собрании сказано:
“От слушания, видения и совершения подношений буддам выростает гора безмерной заслуги.
Все это собрание не исчерпается, пока не будут отброшены все осквернения и страдания Круговорота”.
И далее:
“Любые благопожелания, произнесенные перед ним, осуществятся соответственно” .
В сутре “Объяснение Качеств Поля Манджушри” сказано:
“Все явления обусловлены,
Находятся на острие пожелания.
Какие благопожелания будут сделаны,
Такие плоды и будут обретены”.
Согласно сказанному, с полным пониманием продолжай:
\normalsize
\newpage
\ti
སྙིང་རྗེ་ཆེན་པོས་རྩོད་ལྡན་སྙིགས་མའི་ཞིང་།\\
བཟུང་ནས་སྨོན་ལམ་ཆེན་པོ་ལྔ་བརྒྱ་བཏབ།\\
པད་དཀར་ལྟར་བསྔགས་མཚན་ཐོས་ཕྱིར་མི་ལྡོག\\
སྟོན་པ་ཐུགས་རྗེ་ཅན་ལ་ཕྱག་འཚལ་ལོ།\\
བདག་གཞན་སྒོ་གསུམ་དགེ་ཚོགས་ལོངས་སྤྱོད་བཅས།\\
ཀུན་བཟང་མཆོད་པའི་སྤྲིན་དུ་དམིགས་ནས་འབུལ།\\
ཐོག་མེད་ནས་བསགས་སྡིག་ལྟུང་མ་ལུས་པ། \\
སྙིང་ནས་འགྱོད་པ་དྲག་པོས་སོ་སོར་བཤགས།\\
འཕགས་དང་སོ་སོའི་སྐྱེ་བོའི་དགེ་བ་ནི།\\
དུས་གསུམ་བསགས་ལ་རྗེས་སུ་ཡི་རང་ངོ་།\\
ཟབ་ཅིང་རྒྱ་ཆེ་ཆོས་ཀྱི་འཁོར་ལོའི་ཚུལ། \\
ཕྱོགས་བཅུར་རྒྱུན་མི་འཆད་པར་བསྐོར་དུ་གསོལ།\\
ཁྱོད་ནི་ནམ་མཁའ་ལྟ་བུའི་ཡེ་ཤེས་སྐུ།\\
དུས་གསུམ་འཕོ་འགྱུར་མེད་པར་བཞུགས་མོད་ཀྱི།\\
གདུལ་བྱའི་སྣང་ངོར་སྐྱེ་འཇིག་ཚུལ་སྟོན་ཀྱང་།\\
སྤྲུལ་པའི་གཟུགས་སྐུ་རྟག་ཏུ་སྣང་བར་མཛོད།\\
བདག་གིས་དུས་གསུམ་བསགས་པའི་དགེ་ཚོགས་ཀྱིས། \\
མཁའ་ཁྱབ་འགྲོ་བ་ཀུན་ལ་ཕན་སླད་དུ། \\
ཆོས་ཀྱི་རྒྱལ་པོ་རྟག་ཏུ་མཉེས་བྱེད་ཅིང་།\\
ཆོས་རྗེ་རྒྱལ་བའི་གོ་འཕང་ཐོབ་པར་ཤོག\\
བདག་ཅག་སྙིགས་མའི་འགྲོ་བ་མགོན་མེད་རྣམས།\\
ཐུགས་རྗེས་ལྷག་པར་བཟུང་བའི་བཀའ་དྲིན་ལས།\\
ཞིང་དང་དུས་འདིར་རིན་ཆེན་རྣམ་གསུམ་གྱི། \\
སྣང་བ་ཇི་སྙེད་ཁྱེད་ཀྱི་ཕྲིན་ལས་ཉིད། \\
དེ་ཕྱིར་སྐྱབས་མཆོག་མཚུངས་མེད་གཅིག་པུ་རུ། \\
ཡིད་ཆེས་དད་པས་སྙིང་ནས་གསོལ་འདེབས་ན།\\
སྔོན་གྱི་དམ་བཅའ་ཆེན་པོ་མ་བསྙེལ་བར།\\
བྱང་ཆུབ་བར་དུ་ཐུགས་རྗེས་རྗེས་འཛིན་མཛོད།\\
\\
\ru
Ты с великим состраданием принял мир раздора и вырождений,\\
Высказал пятьсот великих благопожеланий.\\
Тебя восхваляют, как "белый лотос”. \\
Услышать твое имя  — не вернуться в Круговорот.\\
Простираюсь перед милосердным Учителем!\\
Мою и других добродетель трех врат и все богатство,\\
Предлагаю, представив, как облака подношений Самантабхадры.\\
Во всех накопленных е безначальных времен\\
пороках и падениях, без исключения,\\
Чистосердечно признаюсь и сильно раскаиваюсь.\\
Радуюсь собранной в течении трех времен\\
Добродетели благородных и обычных существ.\\
Колесо глубокой и обширной Дхармы\\
Прошу без устали поворачивать в десяти направлениях.\\
Ты — Тело Изначального Знания, подобное небу,\\
Пребываешь неизменным в течении трех времен.\\
Хотя ты являешь перед взором учеников\\
Его рождение и разрушение,\\
Пребывай всегда в форме Нирманакайи!\\
Пусть собрание добродетели,\\
Накопленные мной в течении трех времен\\
Ради блага всех скитальцев, заполняющих небо,\\
Всегда радует Царя Дхармы и\\
Пусть я достигну положения Владыки Дхармы, Победителя.\\
Нас, беззащитных скитальцев эпохи вырождения,\\
Ты по доброте поддерживаешь исключительным милосердием\\
В этом мире в это время все сияние\\
Трех Драгоценностей — твои Деяния.\\
Поэтому к единственному, несравненному, высшему Прибежищу\\
С преданной верой, от всего сердца возношу молитву\\
Не забывай своих великих древних обетов\\
И поддерживай меня милосердной энергией сострадания\\
До достижения Великого Пробуждения!\\
\\
\newpage
\subsection{Рецитация}
\scriptsize
\ti
ཅེས་ཡིད་ཆེས་ཀྱི་དད་པ་དྲག་པོས་སྟོན་པ་དངོས་སུ་བཞུགས་ཡོད་སྙམ་པའི་སྐུ་ལ་རྩེ་གཅིག་དུ་དམིགས་ཏེ།\\
\\
\ru
Так с преданностью и сильной верой, чувствуя, что Учитель действительно присутствует,
однонаправленно (концентрируясь) на образе Тела (Будды, повторяй):\\
\\
\normalsize
\ti
བླ་མ་སྟོན་པ་བཅོམ་ལྡན་འདས་དེ་བཞིན་གཤེགས་པ་དགྲ་བཅོམ་པ་ཡང་དག་པར་རྫོགས་པའི་སངས་རྒྱས་དཔལ་རྒྱལ་བ་ཤཱཀྱ་ཐུབ་པ་ལ་ཕྱག་ཚལ་ལོ། 
།མཆོད་དོ། །སྐྱབས་སུ་མཆིའོ། \\
\\
\ru
ЛАМА ТОНПА ЧОМДЕН ДЕ ДЕБ ЖИН ШЕКПА ДРАЧОМПА ЯНГДАГ ПАР ДЗОКПЕ
САНГГЬЕ ПАЛГЬЯЛВА ШАКьЯ ТУППА ЛА ЧАГ ЦЕЛЛО ЧОТО КЬЯБСУЧИО\\
\\
Простираюсь, подношу и прихожу за Прибежищем к тебе, О, Высший, Учитель, Благостный,
Пришедший к таковости, Повергший врагов, Подлинно и Совершенно Пробужденный,
Славный Победитель, Шакьямуни.\\
\\
\scriptsize
\ti
ཐུགས་རྒྱུད་བསྐུལ་བའི་ཚུལ་དུ་ཤེར་ཕྱིན་ཡི་གེ་ཉུང་ངུ་ལས་གསུངས་པའི་བཟུང་ནི།\\
\\
\ru
Завершив, чтобы призвать поток Ума (Будды), повторяй мантру,
данную в сутре о Запредельном Проникновенном Знании в нескольких словах:\\
\\
\normalsize
\ti
ཏ་དྱ་ཐཱ། ཨོཾ་མུ་ནེ་མུ་ནེ་མ་ཧཱ་མུ་ན་ཡེ་སྭ་ཧཱ།\\
\\
\ru
ТАДЬЯТА ОМ МУНЕ МУНЕ МАХА МУНАЕ СВАХА\\
\\
\scriptsize
\ti ཞེས་ཅི་རིགས་དང་།\\
\\
\ru Затем, начиная с ОМ, повторяй, сколько можешь:\\
\\
\normalsize
\ti ཨོཾ་མུ་ནེ་མུ་ནེ་མ་ཧཱ་མུ་ན་ཡེ་སྭ་ཧཱ།\\
\\
\ru  ОМ МУНЕ МУНЕ МАХА МУНЭЕ СВАХА\\
\\
\newpage
\subsection{Растворение}
\scriptsize
\ti
མན་ཆད་ཅི་འགྲུབ་ཏུ་བཟླའོ། འདི་དག་གི་སྐབས་སུ་སྟོན་པའི་ཡོན་ཏན་རྗེས་སུ་དྲན་ཏེ། དད་པའི་སེམས་ཀྱིས་རྩེ་གཅིག་ཏུ་སྐུ་ཡི་གསལ་སྣང་ལ་དམིགས་ནས། མཚན་བརྗོད་པ་དང་། བཟུང་བཟླས་པའི་རྐྱེན་གྱིས།\\
\ru
Во время этого (повторения) , памятуя о Качествах Учителя, с верой сохраняй
однонаправленную концентрацию на образе сияющего ясного Тела.
Повторяй имена (Будды) и рецитируй мантру, (представляя):\\
\\
\normalsize
\ti
སྟོན་པའི་སྐུ་ལས་ཡེ་ཤེས་ཀྱི་འོད་ཟེར་སྣ་ཚོགས་པའི་སྣང་བ་ཆེན་པོས་བདག་དང་སེམས་ཅན་ཐམས་ཅད་ཀྱི་སྒྲིབ་པ་ཐམས་ཅད་བསལ་ཞིང་།
ཐེག་པ་ཆེན་པོའི་ལམ་གྱི་ཡོན་ཏན་ཚུལ་བཞིན་དུ་སྐྱེས་ཏེ་ཕྱིར་མི་ལྡོག་པའི་ས་ནོན་པར་བསམ།\\
\\
\ru
Из Тела Учителя (исходит) великое сияние разнообразных лучей света Изначального
Знания и очищает полностью мои и всех разумных существ омрачения. Обретаются
соответствующие качества Великой Колесницы, чтобы достигнуть состояния Невозвращающегося.
Так усердствуй.\\
\\
\scriptsize
\ti
བསམ་ལ་དེ་ལྟར་ཅི་ནུས་སུ་བརྩོན་པར་བྱའོ། ཐུན་མཚམས་རྣམས་སུ་མཎྜལ་སོགས་མཆོད་པ་དང་། ཐུབ་སྟོད་ཀྱི་རིགས་དང་། སྙིང་རྗེ་པད་དཀར། རྒྱ་ཆེར་རོལ་པ། སྐྱེས་རབ་སྣ་ཚོགས། དེ་བཞིན་གཤེགས་པའི་མཚན་བརྒྱ་རྩ་བརྒྱད་པ་སོགས་མདོ་གང་འདོད་ཅི་ལྟར་ནུས་པར་བཀླག། དགེ་བའི་རྩ་བ་རྣམས་བླ་མེད་བྱང་ཆུབ་ཏུ་བསྔོ་བ་དང་སྨོན་ལམ་གྱིས་རྒྱས་གདབ་པར་བྱའོ། སྤྱིར་འགྲོ་འཆག་ཉལ་འདུག་གི་སྐབས་ཀུན་ཏུ་སྟོན་པ་ཉིད་མ་བརྗེད་པར་དྲན་པ་དང་། མཚན་མོ་ཡང་སྟོན་པ་དངོས་སུ་བཞུགས་པའི་སྐུ་ཡི་འོད་ཀྱིས་ཕྱོགས་ཐམས་ཅད་ཉིན་མོ་ཤིན་ཏུ་དྭངས་བའི་དུས་ལྟ་བུར་སྣང་བའི་འདུ་ཤེས་ཀྱི་ངང་དུ་གཉིད་ལོག་པར་བྱ། དུས་རྒྱུན་དུ་སྟོན་པ་ཉིད་ཀྱིས་ཇི་ལྟར་ཐུགས་བསྐྱེད་པའི་ཚུལ་ལས་བརྩམ་སྟེ། དུས་གསུམ་གྱི་སངས་རྒྱས་དང་བྱང་ཆུབ་སེམས་དཔའ་ཆེན་པོ་རྣམས་ཀྱི་རྣམ་པར་ཐར་པ་ལ་རྗེས་སུ་གཞོལ་བའི་བྱང་ཆུབ་ཀྱི་སེམས་རིན་པོ་ཆེའི་དམ་བཅའ་ལྷོད་པ་མེད་པའི་ངང་ནས་བྱང་ཆུབ་སེམས་དཔའི་སྤྱོད་པ་སྤྱི་དང་། ཁྱད་པར་ཞི་ལྷག་གི་རྣལ་འབྱོར་ལ་ཅི་ནུས་སུ་བརྩོན་པས་དལ་འབྱོར་ཐོབ་པ་དོན་ལྡན་དུ་འགྱུར་ཏེ། བདག་ཅག་གི་སྟོན་པ་འདི་ཉིད་ཀྱི་མཚན་ཐོས་པ་ཙམ་ཞིག་གིས་རིམ་གྱིས་བྱང་ཆུབ་ཆེན་པོའི་ལམ་ལས་ཕྱིར་མི་ལྡོག་པར་འགྱུར་བ་མདོ་དུ་མ་ནས་གསུངས་ལ། གོང་དུ་བསྟན་པའི་བཟུངས་འདི་ལས་སངས་རྒྱས་ཐམས་ཅད་འབྱུང་ཞིང། གཟུངས་འདི་རྙེད་པའི་མཐུས་ཤཱཀྱའི་རྒྱལ་པོ་ཉིད་སངས་རྒྱས་ཤིང་། སྤྱན་རས་གཟིགས་བྱང་ཆུབ་སེམས་དཔའི་སྤྱོད་པ་མཆོག་རུ་གྱུར་པ་དང་། གཟུངས་འདི་ཐོས་པ་ཙམ་གྱིས་བསོད་ནམས་རྒྱ་ཆེན་པོ་ཚེགས་མེད་པར་འཐོབ་ཅིང་ལས་ཀྱི་སྒྲིབ་པ་ཐམས་ཅད་བྱང་བ་དང་། སྔགས་བསྒྲུབ་པ་ན་བགེགས་མ་མཆིས་པར་གྲུབ་པར་འགྱུར་རོ་ཞེས་ཤེས་རབ་ཀྱི་ཕ་རོལ་ཏུ་ཕྱིན་པ་ཡི་གེ་ཉུང་ངུ་ཞེས་པ་དེ་ཉིད་ལས་གསུངས་ཤིང་། བཀའ་གཞན་ལས་ཀྱང་གཟུངས་འདི་ལན་གཅིག་བཟླས་པས་བསྐལ་པ་བྱེ་བ་ཕྲག་བརྒྱད་ཁྲིའི་བར་དུ་བྱས་པའི་སྡིག་པ་ཐམས་ཅད་བྱང་བར་འགྱུར་པ་སོགས་ཕན་ཡོན་ཚད་མེད་པ་དང་ལྡན་ཞིང། དེ་བཞིན་གཤེགས་པ་ཤཱཀྱ་ཐུབ་པའི་སྙིང་པོ་དམ་པ་ཉིད་དུ་གསུངས་སོ། དད་པ་བསྐྱེད་པ་དང་ཞི་ལྷག་གི་རྣལ་འབྱོར་ལ་ཇི་ལྟར་བརྩོན་པའི་ཚུལ་ཟུར་དུ་བཤད་པར་བྱའོ། །ཞེས་པ་འདི་ནི་བསླབ་གསུམ་ནོར་བུའི་མཛོད་མངའ་དབོན་ཨོ་རྒྱན་བསྟན་འཛིན་ནོར་བུ་ནས་བཀྲ་ཤིས་པའི་ལྷ་རྫས་དང་བཅས་ཏེ་ནན་ཏན་དུ་བསྐུལ་བ་ཡིད་ལ་འཇགས་པའི་སྟེང་དུ་ཉེ་ཆར་ཡང་དབོན་རིན་པོ་ཆེ་ཉིད་ནས་སྤྲུལ་པའི་སྐུ་འཇིགས་མེད་པད་མ་བདེ་ཆེན་ལ་སྦྲན་ཏེ། རིན་ཆེན་དང་པོ་སོགས་བཀྲ་ཤིས་པའི་ལྷ་རྫས་ཀྱི་སྐྱེས་དང་བཅས་མྱུར་དུ་གྲུབ་པར་གྱིས་ཞེས་དམ་པ་ཟུང་གི་བཀས་བསྐུལ་བ་ལ་བརྟེན་ནས། སྟོན་པ་མཆོག་ལ་མི་ཕྱེད་པའི་དད་པ་ཐོབ་ཅིང་། དུས་མཐར་ཆོས་སྨྲ་བའི་མིང་ཙམ་འཛིན་པ ཤཱཀྱའི་རྗེས་འཇུག་མི་ཕམ་འཇམ་དབྱངས་རྒྱ་མཚོས། རྫ་རྡོ་རྗེ་འཕན་ཕྱུག་གི་རི་ཞོལ་ཕུན་ཚོགས་ནོར་བུའི་གླིང་དུ། ཚུལ་འདི་མཐོང་ཐོས་དྲན་རེག་གི་འགྲོ་བ་རྣམས་ཀྱི་རྒྱུད་པ་སྟོན་པ་ཐུབ་པའི་དབང་པོའི་བྱིན་རླབས་མཚུངས་པ་མེད་པ་མངོན་དུ་འཇུག་པར་གྱུར་ཅིག།
\ru
\normalsize
\newpage
\scriptsize
Во время перерывов между сессиями подноси мандалу и прочее,
повторяй “Восхваления Щакьямуни”, “Белый Лотос Сострадания”, “Обширную Игру”,
Джатаки, “Сто восемь имен Татхагаты” и любые другие сутры по желанию.
Посвящай все корни добродетели Высшему Пробуждению и твори благопожелания.
Вообще, при ходьбе, гуляя, лежа и сидя, во всех ситуациях, не забывай думать об Учителе.
Ночью отправляйся спать с ощущением сияния, как во время ясного дня, исходящего
во всех направлениях из Тела действительно присутствующего Учителя. Все время,
подобно тому, как сам Учитель в прошлом зародил Сердце ( Просветления) и следуя
примеру освобождения всех будд и великих бодхисаттв трех времен, в состоянии
нерушимых обетов драгоценной бодхичитты, следуй поведению бодхисаттв в общем и,
в особенности, прилагай как можно больше усилий в йоге Успокоения и Прозрения,
чтобы наполнить смыслом обретенные свободы и преимущества.
Во многих сутрах сказано, что лишь слушание имени Учителя постепешю приведет
на необратимьй путь к великому Пробуждению. В сутре о Запредельном Проникновенном
Знании в нескольких словах сказано:
“Из этой мантры Учителя возникли все Будды. Благодаря силе практики этой мантры
сам царь Шакьев стал Буддой, Авалокита достиг высшего поведения бодхисаттв.
Только слушание этой мантры накапливает обширные заслуги без трудностей,
очищает все кармические омрачения и устраняет все результаты заклинаний”.
В другой сутре сказано: “Лишь одно повторение этой мантры способно очистить
все пороки, накопленные в течении восьмисот биллионов кальп, и приносит другие
безмерные блага.” Сказано, что она -святая сущность Пришедшего к таковости Шакьямуни.
Как зародить веру и практиковать йогу Успокоения и Прозрения, разъяснено в другом месте.
Держатель трех видов обучения Он Орджен Тендзин Норбу, поднеся вещества удачи,
со всей серьезностью попросил меня записать это, поскольку оно уже хранилось
в моем уме. Недавно сам Он Ринпоче через Тулку Джикме Пема Дечена передал
первые драгоценности (золото) и другие вещества удачи и попросил поскорее
завершить (работу). Согласно прошению этих двух святых я, последователь
Шакьяmaмуни) Мипам Джамьянг Джамцо, обретший неизменную веру в Высшего
Учителя, лишь носящий титул проповедника Дхармы в последние времена,
написал это у горы ДЗа Дордже Пенчхук в (монастыре) Пунцхок Норбу Линг
в восьмой день восходящей луны месяца Великих чудес начала благоприятного
года железа-мыши (1899). Да стаяетэто непрекращающимся потоком, приносящим
удивительную пользу Учению и скитальцам. Пусть в потоке (ума) всех скитальдев,
которые видят, слышат, помнят или касаются этой практики прямо войдет несравненная
благодать Учителя, Повелителя Шакьев.\\
\\
Посколъку Просветленное Намерение, деяния, благопожелания, мудрость, любовь и способности
Всех Татхагат и их Сыновей
Есть лишь волшебное проявление непревосходимого Изначального Знания,
Так пусть я и все (скитальцы) достигнут того же.\\
\\
Царь Дхармы, Будда, Бхагавант,
Восемь великих близких Сыновей - бодхисаттв,
Шестнадцать великих благородных Старейших,
Молю, позаботьтесь обо мне, даруйте благодать
Освободите меня от всех болезней, духов и вредоносных влияний.
Пусть я осуществлю великий смысл на благо Учения и скитальцев.
Пусть возрастут качества (моих) переживаний и осознания.
Благословите утвердиться в состоянии окончательного счастья!\\
\\
Написал Мипхам на І2 день 12 месяца года огня - овцы.
\footnotesize